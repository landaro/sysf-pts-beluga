\pdfminorversion=6
\pdfcompresslevel=9
\pdfobjcompresslevel=2

\documentclass[english,pdftex,dvipsnames,leqno,handout]{beamer}% handout option disables overlays
\mode<presentation>{\usetheme{CS-Saar-Uni}}

\usepackage[utf8]{inputenc}
\usepackage{babel}
\usepackage[T1]{fontenc}

\usepackage{epigraph}
\setlength{\epigraphwidth}{0.85\linewidth}
% \setlength{\beforeepigraphskip}{10\baselinskip}
% \setlength{\afterepigraphskip}{2\baselinskip}

% Times/Helvetica/Courier combination
%\usepackage{mathptmx}
%\usepackage[scaled=.90]{helvet}
%\usepackage{courier}

% Palatino/Helvetica/Courier combination
%\usepackage{mathpazo}
%\usepackage[scaled=.95]{helvet}
%\usepackage{courier}

% Disable footnote ruler
\renewcommand{\footnoterule}{}

\usepackage{tikz}
\usetikzlibrary{positioning,calc,decorations.pathreplacing}
\usepackage{calc}
\def\checkmark{\tikz\fill[scale=0.4](0,.35) -- (.25,0) -- (1,.7) -- (.25,.15) -- cycle;}
\def\scalecheck{\resizebox{\widthof{\checkmark}*\ratio{\widthof{x}}{\widthof{\normalsize x}}}{!}{\checkmark}}

\tikzstyle{every picture}+=[remember picture]
\tikzstyle{na} = [baseline=-.5ex]

\newcommand{\tikzmark}[1]{\tikz[overlay,remember picture] \node (#1) {};}

%colors from theme
\definecolor{sbdcyan}{RGB}{026, 058, 107}
\definecolor{sbmcyan}{RGB}{038, 116, 176}
\definecolor{sblcyan}{RGB}{109, 182, 218}
\definecolor{greenblue}{RGB}{0, 127, 79}

% Packages specific to the content
\usepackage{mathpartir}
\usepackage{xspace}
%\usepackage{amsmath}
% \usepackage{amssymb}
% \usepackage{mathpartir}
% \usepackage{ntheorem}

%%% MY MACROS, copy old if needed

% Formatting
\newcommand{\hl}[1]{\emph{\color{sbmcyan} #1}}
\newcommand{\mycite}[1]{{\color{greenblue}\scriptsize[#1]}}

% basic math
\newcommand{\NN}{\ensuremath{\mathbb{N}}}

% spacing
\newcommand{\ms}{\;\,}
\newcommand{\mbin}[1]{\mathbin{\ms #1 \ms}}
\newcommand{\mrel}[1]{\mathrel{\ms #1 \ms}} % meta level binary relation

% meta operators
\newcommand{\mForall}[1]{\forall #1.\ms}
\newcommand{\mExists}[1]{\exists #1.\ms}
\newcommand{\mIff}{\mrel{\iff}}
\newcommand{\mImpl}{\mrel{\Rightarrow}}
\newcommand{\mOf}{\mrel{:}}
\newcommand{\mAnd}{\mrel{\wedge}}
\newcommand{\mOr}{\mrel{\vee}}

% definitions
\newcommand{\eqdef}{\mbin{:=}}
\newcommand{\bnfdef}{\mbin{::=}}

% substitutions
\newcommand{\subst}[1]{\hphantom{|}\![{#1}]}
\newcommand{\id}{\mathsf{id}}
\newcommand{\shift}{\ensuremath{\uparrow}}
\newcommand{\up}{{\Uparrow}}
\newcommand{\scons}{\mathbin{\cdot}}
\newcommand{\scomp}{\mathbin{\circ}}

% adjusting relations
\newcommand{\cons}{\mathbin{\,:\hspace{-0.1em}:\,}}
\newcommand{\rup}[1]{\ensuremath{{#1}^\Uparrow}}
\newcommand{\rext}[1]{\ensuremath{{#1}^{\mathsf{ext}}}}

% systems
\newcommand{\SysF}{\ensuremath{\mathsf{\color{greenblue}F}}}
\newcommand{\SysL}{\ensuremath{{\color{sbmcyan}\lambda2}}}

\newcommand{\ty}{\mathsf{ty}}
\newcommand{\ter}{\mathsf{ter}}

% syntax
\newcommand{\TyF}{\ensuremath{\mathsf{\color{greenblue}Ty_{F}}}}
\newcommand{\TmF}{\ensuremath{\mathsf{\color{greenblue}Tm_{F}}}}

\newcommand{\impf}[2]{\ensuremath{#1 \mathbin{\color{greenblue}\rightarrow} #2}}
\newcommand{\allf}[1]{\ensuremath{{\color{greenblue}\forall.} #1}}
\newcommand{\nallf}[2]{\ensuremath{{\color{greenblue}\forall} #1 {\color{greenblue}.} #2}}
\newcommand{\appf}[2]{\ensuremath{#1 \mathop{\color{greenblue}\$} #2}}
\newcommand{\lamf}[2]{\ensuremath{{\color{greenblue}\lambda} #1 {\color{greenblue}.} #2}}
\newcommand{\tyappf}[2]{\ensuremath{#1 \mathop{\color{greenblue}@} #2}}
\newcommand{\tylamf}[1]{\ensuremath{{\color{greenblue}\Lambda.}#1}}
\newcommand{\ntylamf}[2]{\ensuremath{{\color{greenblue}\Lambda} #1 {\color{greenblue}.} #2}}

\newcommand{\TmL}{\ensuremath{\mathsf{\color{sbmcyan}Tm_{\lambda}}}}

\newcommand{\typl}{\ensuremath{{\color{sbmcyan}\square}}}
\newcommand{\prpl}{\ensuremath{\textrm{\color{sbmcyan}\textasteriskcentered}}}
\newcommand{\appl}[2]{\ensuremath{#1 \mathop{\color{sbmcyan}\$} #2}}
\newcommand{\laml}[2]{\ensuremath{{\color{sbmcyan}\lambda} #1 {\color{sbmcyan}.} #2}}
\newcommand{\prodl}[2]{\ensuremath{{\color{sbmcyan}\Pi} #1 {\color{sbmcyan}.} #2}}

% judgements
\newcommand{\of}{\mathbin{:}}
\newcommand{\tsf}{\mathrel{\color{greenblue}\vdash}}
\newcommand{\tsl}{\mathrel{\color{sbmcyan}\vdash}}

\newcommand{\istyf}[1]{\ensuremath{#1 \ms \mathbf{\color{greenblue}ty}}}
\newcommand{\typingf}[2]{\ensuremath{#1 \mathrel{\color{greenblue}\of_{\mathsf{F}}} #2}}

\newcommand{\univl}[1]{\ensuremath{\mathcal{\color{sbmcyan}U}\, #1}}
\newcommand{\typingl}[2]{\ensuremath{#1 \mathrel{\color{sbmcyan}\of_{2}} #2}}

\newcommand{\tyrel}[2]{\ensuremath{#1 \mathrel{\sim} #2}}
\newcommand{\tmrel}[2]{\ensuremath{#1 \mathrel{\approx} #2}}

% relational context morphisms
\newcommand{\tyctxrelFL}[3]{\ensuremath{#1\mathrel{\mathop{\longrightarrow}^{#2}\limits}#3}}
\newcommand{\tyctxrelLF}[3]{\ensuremath{#1\mathrel{\mathop{\longleftarrow}^{#2}\limits}#3}}
\newcommand{\tmctxrelFL}[4]{\ensuremath{#1\mathrel{\mathop{\longrightarrow}^{#2}_{#3}\limits}#4}}
\newcommand{\tmctxrelLF}[4]{\ensuremath{#1\mathrel{\mathop{\longleftarrow}^{#2}_{#3}\limits}#4}}

% LambdaProlog symbols
\newcommand{\lpProp}{\ensuremath{\mathbf{o}}}
\newcommand{\lpPi}[1]{\boldsymbol{\Pi} #1.\ms}
\newcommand{\lpApp}[2]{#1\langle#2\rangle}
\newcommand{\lpImp}{\mrel{=\!\blacktriangleright}}

% G Symbols
\newcommand{\gProp}{\ensuremath{\mathbf{Prop}}}


%%% END MACROS

% Generate Section Title Pages
% \AtBeginSection{\frame[noframenumbering]{\sectionpage}}
% \setbeamertemplate{section page}
% {
%   \begin{centering}
%     \begin{beamercolorbox}[sep=4pt,center]{part title}
%       \usebeamerfont{section title}\insertsection\par
%     \end{beamercolorbox}
%   \end{centering}
% }

%%%%%%%%%%%%%%%%%%%%%%%%%%%%%%%%%%%%%%%%%%%%%%%%%%%%%%%%%%%%%%%%%%%%%%%%%%%%%%%%
% Metadata
%%%%%%%%%%%%%%%%%%%%%%%%%%%%%%%%%%%%%%%%%%%%%%%%%%%%%%%%%%%%%%%%%%%%%%%%%%%%%%%%
\title[$F$ and $\lambda 2$ -- A Case Study]{Relating System F and \SysL: \\A Case Study in Coq, Abella and Beluga}
%\subtitle{}

\author[Jonas Kaiser]{
  \texorpdfstring{
    \href{http://www.ps.uni-saarland.de/~jkaiser}{\underline{Jonas Kaiser}} \and
    \href{http://www.cs.mcgill.ca/~bpientka}{Brigitte Pientka} \and
    \href{http://www.ps.uni-saarland.de/~smolka}{Gert Smolka}
  }
  {Jonas Kaiser}}

\institute[Saarland University]{\normalsize FSCD 2017, Oxford}

\date{September 4, 2017}

%%%%%%%%%%%%%%%%%%%%%%%%%%%%%%%%%%%%%%%%%%%%%%%%%%%%%%%%%%%%%%%%%%%%%%%%%%%%%%%%
% Content
%%%%%%%%%%%%%%%%%%%%%%%%%%%%%%%%%%%%%%%%%%%%%%%%%%%%%%%%%%%%%%%%%%%%%%%%%%%%%%%%

\begin{document}

\section*{Introduction}

\begin{frame}[plain]
  \titlepage
\end{frame}

\begin{frame}
  \frametitle{Overview}
  \tableofcontents
\end{frame}

\section{Background \& Motivation}

\begin{frame}
  \frametitle{Introduction}
  \begin{block}{1) The Mathematical Problem}
    \begin{itemize}
    \item Several variants of System F \mycite{Girard '72} / PTLC \mycite{Reynolds '74} exist.
    \item In particular
      \begin{itemize}
      \item $\SysF$ -- two-sorted with explicit type variable context \mycite{Harper '13}
      \item $\SysL$ -- single-sorted pure type system (PTS) style \mycite{Barendregt '91}
      \end{itemize}
    \item Transport of results relies on correspondence, e.g.\ \hl{reduction of typing}.
    \end{itemize}
  \end{block}
  \begin{block}{2) The Formalisation Problem}
    \begin{itemize}
    \item Syntax with binders.
    \item Representation of variables.
    \item Tracking of contextual information.
    \end{itemize}
  \end{block}
\end{frame}

\begin{frame}
  \frametitle{Syntactic Variants $\SysF$ and $\SysL$  (Named)}
  Two-sorted, non-uniform, type formation, typing
  \begin{align*}
    &\TyF & A, B \bnfdef &X \mid \impf{A}{B} \mid \nallf{X}{A} \\
    &\TmF & s, t \bnfdef &x \mid \appf{s}{t} \mid \lamf{x \of A}{s} \mid \tyappf{s}{A} \mid \ntylamf{X}{s}\\[.6em]
    & & &\Delta \tsf \istyf{A} \\
    & & &\Delta; \Gamma \tsf \typingf{s}{A} \\
    \intertext{Single-sorted, uniform, typing}
    &\TmL & a, b \bnfdef &x \mid \prpl \mid \typl \mid \appl{a}{b} \mid \laml{x \of a}{b} \mid \prodl{x \of a}{b}\\[.6em]
    & & & \Psi \tsl \typingl{a}{b}
  \end{align*}
\end{frame}

\begin{frame}
  \frametitle{Main Goal}
  Reduction from $\SysF$ to $\SysL$:
  \begin{align*}
    {} \tsf \istyf{A} &\mIff \mExists a \tyrel{A}{a} \mAnd {} \tsl \typingl{a}{\prpl}\\
    {} \tsf \typingf{s}{A} &\mIff \mExists{b,a} \tmrel{s}{b} \mAnd \tyrel{A}{a} \mAnd {} \tsl \typingl{b}{a} \mAnd {} \tsl \typingl{a}{\prpl}
  \end{align*}
  Reduction from $\SysL$ to $\SysF$:
  \begin{align*}
    {} \tsl \typingl{a}{\prpl} &\mIff \mExists A \tyrel{A}{a} \mAnd {} \tsf \istyf{A}\\
    {} \tsl \typingl{b}{a} \mAnd {} \tsl \typingl{a}{\prpl} &\mIff \mExists {s,A} \tmrel{s}{b} \mAnd \tyrel{A}{a} \mAnd {} \tsf \typingf{s}{A}\\
  \end{align*}
  \begin{center}
    \hl{Question:}\\How are $\sim$ and $\approx$ defined/established?
  \end{center}
\end{frame}


\begin{frame}
  \frametitle{Related Work}
  \begin{itemize}
  \item The reduction result is partially discussed in \mycite{Geuvers '93}.
    \begin{itemize}
    \item Primarily argues the forward preservation of typing.
    \item The syntactic correspondence ($\sim$/$\approx$) is left implicit.
    \end{itemize}
  \item We give a Coq formalisation of the full reduction in \mycite{K/Tebbi/Smolka '17}.
    \begin{itemize}
    \item Pairs of translation functions establish the syntactic correspondence.
    \item Requires involved cancellation laws.
    \item Proofs based on an extension of context morphism lemmas \mycite{Goguen/McKinna '97, Adams '06}.
    \end{itemize}
  \end{itemize}
\end{frame}

\section{The Equivalence Proof}

\begin{frame}
  \frametitle{Main Challenges of the Proof}
  \begin{itemize}
  \item Different expressivity prior to typing.
  \item Disambiguation of unified PTS syntax, e.g.\
    \begin{align*}
      \tyrel{\impf{A}{B}}{\prodl{a}{b}} &\qquad\textup{or}\qquad \tyrel{\allf{B}}{\prodl{a}{b}}\\
      \tmrel{\tyappf{s}{B}}{\appl{a}{b}} &\qquad\textup{or}\qquad \tmrel{\appf{s}{t}}{\appl{a}{b}}
    \end{align*}
    Depends on typing information\ldots
  \item Mismatch of binding structures.
    \begin{itemize}
    \item Some PTS products are vacuous.
    \item A PTS abstraction corresponds either to a term- or a type-abstraction.
    \end{itemize}
  \end{itemize}
\end{frame}

\begin{frame}
  \frametitle{Overview of the Proof}
  \begin{itemize}
  \item \hl{Step 1:} Define $\sim$ and $\approx$ as inductive relations.
  \item \hl{Step 2:} Establish the following \hl{six properties} of $\sim$ and $\approx$:
    \begin{enumerate}
    \item $\sim$ is functional and injective.
    \item $\sim$ is L-total and type-formation preserving on well-formed types of $\SysF$.
    \item $\sim$ is R-total and type-formation preserving on propositions of $\SysL$.
    \item $\approx$ is functional and injective.
    \item $\approx$ is L-total and typing preserving on well-typed terms of $\SysF$.
    \item $\approx$ is R-total and typing preserving on proofs of $\SysL$.
    \end{enumerate}
  \item \hl{Step 3:} Derive from these the four equivalences that together establish the reduction result.
  \end{itemize}
  \begin{center}
    \hl{Observation:}\\
    L/R-totality are only required for a sensible fragment of the respective language.
    On these fragments we effectively\\ obtain a 1--1 correspondence.
  \end{center}
\end{frame}

\begin{frame}
  \frametitle{Relating the Types, $\tyrel{A}{a}$}
  \begin{mathpar}
    \inferrule*{(X,y) \in \Theta}{\Theta \vdash \tyrel{X}{y}} \\
    \inferrule*[right=$y \notin \Theta$]{\Theta \vdash \tyrel{A}{a} \\ \Theta \vdash \tyrel{B}{b}}{\Theta \vdash \tyrel{\impf{A}{B}}{\prodl{y \of a}{b}}} \\
    \inferrule*[right={$X,y \notin \Theta$}]{\Theta, (X,y) \vdash \tyrel{A}{a}}{\Theta \vdash \tyrel{\nallf{X}{A}}{\prodl{y \of \prpl}{a}}}
  \end{mathpar}
\end{frame}

\begin{frame}
  \frametitle{Relating the Terms, $\tmrel{s}{b}$}
  \begin{mathpar}
    \inferrule*{(x,y) \in \Sigma}{\Theta;\Sigma \vdash \tmrel{x}{y}} \\
    \inferrule*{\Theta;\Sigma \vdash \tmrel{s}{a} \\ \Theta;\Sigma \vdash \tmrel{t}{b}}{\Theta;\Sigma \vdash \tmrel{\appf{s}{t}}{\appl{a}{b}}} \and
    \inferrule*{\Theta;\Sigma \vdash \tmrel{s}{a} \\ \Theta \vdash \tyrel{A}{b}}{\Theta;\Sigma \vdash \tmrel{\tyappf{s}{A}}{\appl{a}{b}}} \\
    \inferrule*[right={$x,y \notin \Theta,\Sigma$}]{\Theta \vdash \tyrel{A}{a} \\ \Theta;\Sigma, (x,y) \vdash \tmrel{s}{b}}{\Theta;\Sigma \vdash \tmrel{\lamf{x \of A}{s}}{\laml{y \of a}{b}}} \\
    \inferrule*[right={$X,y \notin \Theta,\Sigma$}]{\Theta, (X,y);\Sigma \vdash \tmrel{s}{a}}{\Theta;\Sigma \vdash \tmrel{\ntylamf{X}{s}}{\laml{y \of \prpl}}{a}}
  \end{mathpar}
\end{frame}


\begin{frame}
  \frametitle{The Proof}
  \begin{itemize}
  \item The six properties:
    \begin{itemize}
    \item Rely on meta-theoretic properties, e.g.:
      \begin{itemize}
      \item Propagation: $\Delta,\Gamma \tsf \typingf{s}{A} \mImpl \Delta \tsf \istyf{A}$.
      \item Degeneracy of $\typl$: $\Psi \tsl \typingl{a}{\typl} \mImpl a = \prpl$.
      \end{itemize}
    \item Interaction of various contexts: $\Delta, \Gamma, \Psi, \Theta, \Sigma$.
    \item The direction $\SysL \leadsto \SysF$ is harder, as structure has to be recovered.
    \end{itemize}
  \item Deriving the equivalences, e.g.:
    \begin{align*}
      {} \tsf \istyf{A} &\mIff \mExists a \tyrel{A}{a} \mAnd {} \tsl \typingl{a}{\prpl}
    \end{align*}
    \vspace{-1.4em}
    \begin{itemize}
    \item Proof ($\Rightarrow$): Immediate from L-totality and preservation of $\sim$.
    \item Proof ($\Leftarrow$): R-totality and preservation of $\sim$ on ${} \tsl \typingl{a}{\prpl}$ yields an $A'$ with $\tyrel{A'}{a}$ and ${} \tsf \istyf{A'}$.
      Injectivity of $\sim$ entails $A = A'$.
    \end{itemize}
  \item The other three equivalences are similar.
  \end{itemize}
\end{frame}

\section{Formalisation}

\begin{frame}
  \frametitle{Formalising the Proof}
  \structure{How \ldots}
  \begin{itemize}
  \item \ldots do we formally represent the syntax and judgements?
  \item \ldots do we manage local variable binding?
  \item \ldots track various pieces of contextual information?
  \end{itemize}
  \structure{In various proof assistants:}
  \begin{itemize}
  \item Coq -- mature, general purpose, interactive, constructive type theory
  \item Abella -- special purpose, interactive, two-level logic
  \item Beluga -- dep.\ typed programming, contextual modal type theory
  \end{itemize}
\end{frame}

\subsection{de Bruijn -- Coq}

\begin{frame}
  \begin{center}
    \begin{Large}
      \hl{-- Coq --\\[1em]An Index For An Index}
    \end{Large}\\[2em]
    de Bruijn, Parallel Substitutions, DIY Invariants
  \end{center}
\end{frame}

\begin{frame}
  \frametitle{Coq -- Representation}
  \begin{itemize}
  \item First order de Bruijn encoding, e.g.:
    \begin{align*}
      A, B \bnfdef &n_\ty \mid \impf{A}{B} \mid \allf{A} & &n \in \NN \\
      s, t \bnfdef &n_\ter \mid \appf{s}{t} \mid \lamf{A}{s} \mid \tyappf{s}{A} \mid \tylamf{s} & &
    \end{align*}
  \item Typing contexts:
    \begin{align*}
      \Delta &\mOf \NN & &\textup{-- \hl{excl.\ upper bound for free type vars}}\\
      \Gamma &\mOf \mathsf{list}\;\TyF & &\textup{-- \hl{dangling term vars reference by position}}
    \end{align*}
  \item Parallel substitutions $\sigma \of \NN \to \mathcal{T}$:
    \begin{align*}
      (\allf{A})\subst{\sigma} & \leadsto \allf{A\subst{\up\sigma}} & \up\sigma &\eqdef 0_\ty \scons \sigma \scomp \shift
    \end{align*}
  \item Instantiation generated with \hl{Autosubst} library \mycite{Schäfer/Tebbi/Smolka '15}.
  \end{itemize}
\end{frame}

\begin{frame}
  \frametitle{Coq -- Relating Indices}
  \begin{itemize}
  \item Relating open terms requires tracking of related indices:
    \begin{align*}
      R, S \of \mathsf{list}\;(\NN \times \NN)
    \end{align*}
  \item Traversal of binders requires context adjustments:
    \begin{mathpar}
      \inferrule*{R \vdash \tyrel{A}{a} \\ {\color{red}\rup{R}} \vdash \tyrel{B}{b}}{R \vdash \tyrel{\impf{A}{B}}{\prodl{a}{b}}} \and
      \inferrule*{{\color{red}\rext{R}} \vdash \tyrel{A}{a}}{R \vdash \tyrel{\allf{A}}{\prodl{\prpl}{a}}}
    \end{mathpar}
    \begin{align*}
      {\color{red}\rext{R}} &\eqdef (0,0) \cons \mathsf{map}\,\,(\shift \times \shift)\,\,R\\
      {\color{red}\rup{R}} &\eqdef \mathsf{map}\,\,(\id\, \times \shift)\,\,R\\
    \end{align*}
    \vspace{-2em}
    \begin{mathpar}
      \inferrule*{R \vdash \tyrel{A}{a} \\ {\color{red}\rup{R};\rext{S}} \vdash \tmrel{s}{b}}{R;S \vdash \tmrel{\lamf{A}{s}}{\laml{a}{b}}}
    \end{mathpar}
  \end{itemize}
\end{frame}

\begin{frame}
  \frametitle{Coq -- Context Managment, Proofs of P1 and P4.}
  \begin{itemize}
  \item $\rext{R}$ and $\rup{R}$ preserve injectivity and functionality of $R$.
  \item Inj/func of $R$ entails  inj/func of $R \vdash \tyrel{\_}{\_}$. \hl{(P1)}
  \item Func of $R, S$ entails func of $R;S \vdash \tmrel{\_}{\_}$. \hl{(P4a)}
  \item Write $R \parallel S$ for $R$ and $S$ having disjoint ranges.
    \begin{itemize}
    \item W.l.o.g.: $R \parallel S \mImpl \rup{R} \parallel \rext{S}$.
    \item $R \parallel S \mImpl \neg (R \vdash \tyrel{A}{a} \mAnd R;S \vdash \tmrel{s}{a})$.
    \end{itemize}
  \item Inj of $R,S$ and $R \parallel S$ entail inj of $R;S \vdash \tmrel{\_}{\_}$. \hl{(P4b)}
  \end{itemize}
\end{frame}

\begin{frame}
  \frametitle{Coq -- Proof of P2, custom invariants}
  \begin{itemize}
  \item P2: L-totality and preservation of type formation for $\sim$
  \item Recall the core idea of \hl{context morphism lemmas}:
    \begin{enumerate}
    \item Specify desired property on variables of initial context.
    \item Prove closure of property under context extensions.
    \item Lift property inductively to judgements.
    \end{enumerate}
  \item Let $\tyctxrelFL{\Delta}{R}{\Psi} \eqdef \mForall{x < \Delta} \mExists y (x,y) \in R \mAnd (\typingl{y}{\prpl}) \mathrel{\in_\lambda} \Psi$.
    \begin{align*}
      \tyctxrelFL{\Delta}{R}{\Psi} &\mImpl \tyctxrelFL{\Delta}{\rup{R}}{\Psi,a} & &\textup{(ext.\ with new term var)}\\
      \tyctxrelFL{\Delta}{R}{\Psi} &\mImpl \tyctxrelFL{\Delta+1}{\rext{R}}{\Psi,\prpl} & &\textup{(ext.\ with new type var)}
    \end{align*}
  \item Prove \hl{(P2)} by induction on $\Delta \tsf \istyf{A}$:
    \begin{align*}
      \Delta \tsf \istyf{A} \mImpl \mForall {R,\Psi} \tyctxrelFL{\Delta}{R}{\Psi} \mImpl \mExists a R \vdash \tyrel{A}{a} \mAnd \Psi \tsl \typingl{a}{\prpl}
    \end{align*}
  \item Repeat similarly for \hl{(P3)}, \hl{(P5)} and \hl{(P6)}.
  \end{itemize}
\end{frame}

\subsection{HOAS/$\nabla$  -- Abella}

\begin{frame}
  \begin{center}
    \begin{Large}
      \hl{-- Abella --\\[1em]Inversions Galore!}
    \end{Large}\\[2em]
    HOAS, two-level logic, $\nabla$-quantification, relational proof search
  \end{center}
\end{frame}

\begin{frame}
  \frametitle{Abella \mycite{Miller, Chaudhuri et al.\ '14}}
  \begin{itemize}
  \item Two-level logic:
    \begin{itemize}
    \item \hl{Specification Level:} $\lambda$Prolog, HOAS, logic predicates, proof search
      \begin{align*}
        &\lamf{\_}{\_} \mOf \TyF \to (\TmF \to \TmF) \to \TmF\\
        &\typingf{\_}{\_} \mOf \TmF \to \TyF \to \lpProp
      \end{align*}
    \item \hl{Reasoing Level:} $\mathcal{G}$ -- int.\ pred.\ frag.\ of STT with $\nabla$-quantification
      \begin{align*}
        \nabla x.\ms \nabla y. \ms x \neq y
      \end{align*}
    \end{itemize}
  \item Logical Embedding $\{\_ \vdash \_\} \mOf [\lpProp] \to \lpProp \to \gProp$
    \begin{itemize}
    \item $\{L \vdash J\}$ holds in $\mathcal{G}$ iff $J$ has a $\lambda$Prolog-derivation, given hypotheses $L$.
    \item Exposes inductive structure of derivations.
    \end{itemize}
  \end{itemize}
\end{frame}

\begin{frame}
  \frametitle{Abella - Syntactic Relations as Logical Predicates}
  \begin{align*}
    \tyrel{\_}{\_} &\mOf \TyF \to \TmL \to \lpProp & \tmrel{\_}{\_} &\mOf \TmF \to \TmL \to \lpProp
  \end{align*}
  \begin{mathpar}
    \inferrule*{\tyrel{A}{a} \\ \lpPi {x} \tyrel{B}{\lpApp{b}{x}}}{\tyrel{\impf{A}{B}}{\prodl{a}{b}}} \and
    \inferrule*{\lpPi {x\,y} \tyrel{x}{y} \lpImp \tyrel{\lpApp{A}{x}}{\lpApp{a}{y}}}{\tyrel{\allf{A}}{\prodl{\prpl}{a}}}
  \end{mathpar}
  \begin{mathpar}
    \inferrule*{\tmrel{s}{a} \\ \tmrel{t}{b}}{\tmrel{\appf{s}{t}}{\appl{a}{b}}}\and
    \inferrule*{\tyrel{A}{a} \\ \lpPi {x\,y} \tmrel{x}{y} \lpImp \tmrel{\lpApp{s}{x}}{\lpApp{b}{y}}}{\tmrel{\lamf{A}{s}}{\laml{a}{b}}} \\
    \inferrule*{\tmrel{s}{a} \\ \tyrel{B}{b}}{\tmrel{\tyappf{s}{B}}{\appl{a}{b}}}\and
    \inferrule*{\lpPi {x\,y} \tyrel{x}{y} \lpImp \tmrel{\lpApp{s}{x}}{\lpApp{b}{y}}}{\tmrel{\tylamf{s}}{\laml{\prpl}{b}}}
  \end{mathpar}
\end{frame}

\begin{frame}
  \frametitle{Abella - Embedding, Instantiation \& Cut}
  \structure{Embedding example (recall premise of rule for \tmrel{\tylamf{s}}{\laml{\prpl}{b}}):}
  \begin{align*}
    &\{ L \vdash \lpPi {x\,y} \tyrel{x}{y} \lpImp \tmrel{\lpApp{s}{x}}{\lpApp{b}{y}}\} \\
    \leadsto \quad &\nabla x,y. \{ L \vdash \tyrel{x}{y} \lpImp \tmrel{\lpApp{s}{x}}{\lpApp{b}{y}}\}\\
    \leadsto \quad &\nabla x,y. \{ L, \tyrel{x}{y} \vdash \tmrel{\lpApp{s}{x}}{\lpApp{b}{y}}\} \\
    \leadsto \quad &\{ L, \tyrel{n_1}{n_2} \vdash \tmrel{\lpApp{s}{n_1}}{\lpApp{b}{n_2}}\}
  \end{align*}
  \structure{Reasoning in $\mathcal{G}$ with Instantiation \& Cut:}
  \begin{mathpar}
    \inferrule*[right=cut]{\{ L \vdash \tyrel{A}{a}\} \\ \inferrule*[right=inst]{\{ L, \tyrel{n_1}{n_2} \vdash  \tmrel{\lpApp{s}{n_1}}{\lpApp{b}{n_2}}\}}{\{ L, \tyrel{A}{a} \vdash \tmrel{\lpApp{s}{A}}{\lpApp{b}{a}}\}}}{\{ L \vdash \tmrel{\lpApp{s}{A}}{\lpApp{b}{a}}\}}
  \end{mathpar}
\end{frame}

\begin{frame}
  \frametitle{Abella -- Context Management}
  \begin{itemize}
  \item Contexts $L \of [\lpProp]$ are lists of arbitrary logical predicate instances.
  \item The embedding has a backchaining rule \hl{$J \in L \mImpl \{L \vdash J\}$}.
  \item We want typing/relational contexts that only contain information about variables, i.e.\ \hl{nominals}. $\Rightarrow$ Inductive $\mathcal{G}$-predicates:
    {\color{greenblue}\begin{align*}
                        \mathsf{Define}\; &C_\approx \of [\lpProp] \to \gProp\;\mathsf{by}\\
                                          &C_\approx(\bullet);\\
                                          &\nabla x\,y,\ms C_\approx(L, \tyrel{x}{y}) \eqdef C_\approx(L);\\
                                          &\nabla x\,y,\ms C_\approx(L, \tmrel{x}{y}) \eqdef C_\approx(L).
    \end{align*}}\vspace{-1.6em}
    \item Use {\color{greenblue}$C_\approx(L)$} to establish several inversion lemmas that discard spurious instances of backchaining. Required for \hl{(P1)} and \hl{(P4)}.
    \item \hl{Note 1:} {\color{greenblue}$C_\approx(L)$} constrains $L$ to exactly track related variables.
    \item \hl{Note 2:} {\color{greenblue}$C_\approx(L)$} forces $L$ to be injective, functional \& range-disjoint.
  \end{itemize}
\end{frame}

\begin{frame}
  \frametitle{Abella -- Proof of P2, More Inversions}
  \begin{itemize}
  \item L-totality \& preservation of type formation \hl{(P2)}:
    \begin{align*}
      \{L_F \vdash \istyf{A}\} \mImpl \mForall {L_\approx \, L_2} &C_R(L_F \mid L_\approx \mid L_2) \mImpl\\
      & \mExists {a} \{L_\approx \vdash \tyrel{A}{a}\} \mAnd \{L_2 \vdash \typingl{a}{\prpl}\}
    \end{align*}
    \begin{mathpar}
      \inferrule*{~}{C_R(\bullet \mid \bullet \mid \bullet)} \and
      \inferrule*{C_R(L_F \mid L_\approx \mid L_2) \\ {\color{sblcyan}x,y\;\textup{fresh for}\;L_F,L_\approx,L_2}}{C_R(L_F, \istyf{x} \mid L_\approx, \tyrel{x}{y} \mid L_2,\typingl{y}{\prpl})} \\
      \inferrule*{\{L_F \vdash \istyf{A}\} \\ \{L_\approx \vdash \tyrel{A}{a} \}\\ \{L_2 \vdash \typingl{a}{\prpl}\} \\\\ C_R(L_F \mid L_\approx \mid L_2) \\ {\color{sblcyan}x,y\;\textup{fresh for}\;L_F,L_\approx,L_2,A,a}}{C_R(L_F, \typingf{x}{A} \mid L_\approx, \tmrel{x}{y} \mid L_2, \typingl{y}{a})}
    \end{mathpar}
  \item $C_R(L_F \mid L_\approx \mid L_2)$ ensures that $L_\approx$ precisely relates $L_F$ and $L_2$, e.g.:
    \begin{align*}
      C_R(L_F \mid & L_\approx \mid L_2) \mImpl \tyrel{x}{y} \in L_\approx \mImpl \\
      &\istyf{x} \in L_F \mAnd \typingl{y}{\prpl} \in L_2 \mAnd \textup{"$x$ and $y$ are \hl{nominals}"}
    \end{align*}
  \end{itemize}
\end{frame}

\subsection{HOAS/CMTT -- Beluga}
% special remark:  context schemas not always inferrable, due to rather specific requirements (e.g. x : *), contrary to common believe.

\begin{frame}
  \begin{center}
    \begin{Large}
      \hl{-- Beluga --\\[1em]Give Me Some (Specific) Context, Please!}
    \end{Large}\\[2em]
    HOAS, contextual modal type theory, context schemas, proof terms
  \end{center}
\end{frame}


\section*{Conclusion}

\begin{frame}
  \frametitle{Adequacy / Faithfulness of Representation}
  \begin{itemize}
  \item De Bruijn -- \hl{a non-issue}
    \begin{itemize}
    \item First-order clean embedding.
    \item All three proof assistants internally implemented using de Bruijn.
    \item Canonical implementation of the Barendregt convention.
    \end{itemize}
  \item HOAS -- has to be argued externally
    \begin{itemize}
    \item Object and host abstractions/function spaces intertwined.
    \item Why is this a faithful representation?
    \item Well-understood for basic HOAS
    \end{itemize}
  \item HOAS with extras -- not as obvious \ldots
    \begin{itemize}
    \item Abella -- nominal infrastructure
    \item Beluga -- first class contexts
    \end{itemize}
  \item We obtain expected properties and the same coarse proof structure works in all three cases. \hl{$\Rightarrow$ A certain degree of trust.}
  \end{itemize}
\end{frame}

\begin{frame}
  \frametitle{Conclusion}
  \begin{itemize}
  \item Summary:
    \begin{itemize}
    \item Result: reduction of typing for two variants of System F.
    \item Formalised using three different approaches: first-order de Bruijn, HOAS with nominals, HOAS with $1^{st}$-class contexts
    \end{itemize}
  \item Future Work:
    \begin{itemize}
    \item Consider the same result for STLC and F$_\omega$.
    \item Consider a correspondence of the computational behaviour.
    \item Consider other formalisation techniques, e.g.: LN \mycite{Aydemir et al. '08}, HYBRID \mycite{Capretta/Felty '06} (both Isabelle and Coq), Twelf, \ldots
    \end{itemize}
  \item A new benchmark? Complements and extends
    \begin{itemize}
    \item POPLmark -- larger case studies but only considering a single system
    \item ORBI -- considers multi system settings, but only very small examples
    \item Here -- multi system, complex contextual information, several binders
    \end{itemize}
  \end{itemize}
\end{frame}

\begin{frame}
  \frametitle{The Take-Home Lesson}
  \begin{itemize}
  \item Syntax with local binding and complex contextual information is tricky.
  \item \hl{There is no silver bullet!}
  \item Certain techniques go well together:
    \begin{itemize}
    \item De Bruijn/parallel substitutions/CML-style invariants.
    \item HOAS with context constraints/schemas and corresponding inversions.
    \item Relations capture correspondences which hold on language fragments.
    \end{itemize}
  \item Doing the proof three times provides a lot of mathematical insight:
    \begin{itemize}
    \item Each system imposes technical complications.
    \item But certain issues arise in all variants, namely \ldots
    \item \ldots splitting of the PTS variable scope,
    \item \ldots disambiguation of PTS applications,
    \item \ldots maintaining related pieces of contextual information.
    \end{itemize}
  \end{itemize}
\end{frame}

\begin{frame}
  \begin{center}
    \begin{Large}
      \hl{Thank you for your attention.}
    \end{Large}
    \vfill
    \url{http://www.ps.uni-saarland.de/extras/fscd17/}
  \end{center}
\end{frame}

%%%%%%%%%%%%%%%%%%%%%%%%%%%%%%%%%%%%%%%%%%%%%%%%%%%%%%%%%%%%%%%%%%%%%%%%%%%%%%%%
% Backup Slides
%%%%%%%%%%%%%%%%%%%%%%%%%%%%%%%%%%%%%%%%%%%%%%%%%%%%%%%%%%%%%%%%%%%%%%%%%%%%%%%%

\section*{Backup}


\end{document}

%%% Local Variables:
%%% mode: latex
%%% TeX-master: t
%%% End:
