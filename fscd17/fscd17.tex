\documentclass[a4paper,UKenglish]{lipics-v2016}

\usepackage{microtype}%if unwanted, comment out or use option "draft"

% custom packages
\usepackage{mathpartir}
\usepackage{xspace}

%\graphicspath{{./graphics/}}%helpful if your graphic files are in another directory

\bibliographystyle{plainurl}% the recommended bibstyle

% Author macros::begin %%%%%%%%%%%%%%%%%%%%%%%%%%%%%%%%%%%%%%%%%%%%%%%%
\title{Relating System F and $\lambda$2: A Case Study in Coq, Abella and Beluga}
\titlerunning{Relating System F and $\lambda2$}

%% Please provide for each author the \author and \affil macro, even when authors have the same affiliation, i.e. for each author there needs to be the  \author and \affil macros
\author[1]{Jonas Kaiser}
\author[2]{Brigitte Pientka}
\author[1]{Gert Smolka}
\affil[1]{Saarland University, Saarbrücken, Germany\\
  \texttt{\{jkaiser,smolka\}}@ps.uni-saarland.de}
\affil[2]{school, city, country\\
  \texttt{MAIL}}
\authorrunning{J. Kaiser, B. Pientka and G. Smolka} %mandatory. First: Use abbreviated first/middle names. Second (only in severe cases): Use first author plus 'et. al.'

\Copyright{Jonas Kaiser, Brigitte Pientka and Gert Smolka}%mandatory, please use full first names. LIPIcs license is "CC-BY";  http://creativecommons.org/licenses/by/3.0/

\subjclass{F.4.1 Mathematical Logic -- Lambda calculus and related systems}% mandatory: Please choose ACM 1998 classifications from http://www.acm.org/about/class/ccs98-html . E.g., cite as "F.1.1 Models of Computation".
\keywords{Pure Type Systems, System F, de Bruijn Syntax, Higher-Order Abstract Syntax, Contextual Reasoning}% mandatory: Please provide 1-5 keywords
% Author macros::end %%%%%%%%%%%%%%%%%%%%%%%%%%%%%%%%%%%%%%%%%%%%%%%%%

%Editor-only macros:: begin (do not touch as author)%%%%%%%%%%%%%%%%%%%%%%%%%%%%%%%%%%
\EventEditors{John Q. Open and Joan R. Acces}
\EventNoEds{2}
\EventLongTitle{2nd International Conference on Formal Structures for Computation and Deduction (FSCD 2017)}
\EventShortTitle{FSCD 2017}
\EventAcronym{FSCD}
\EventYear{2017}
\EventDate{September 3--9, 2017}
\EventLocation{Oxford, United Kingdom}
\EventLogo{}
\SeriesVolume{2}
\ArticleNo{XY}
% Editor-only macros::end %%%%%%%%%%%%%%%%%%%%%%%%%%%%%%%%%%%%%%%%%%%%%%%


%%% Content macros::begin %%%%%%%%%%%%%%%%%%%%%%%%%%%%%%%%%%

% uniform meta space
\newcommand{\ms}{\,}
\newcommand{\mrel}[1]{\mathrel{\ms #1 \ms}}

% Meta-Level Propositions
\newcommand{\Prop}{\ensuremath{\mathsf{PROP}}}
\newcommand{\Nat}{\mathbb{N}}

% Meta-level Symbols and Operators
\newcommand{\dom}[1]{\ensuremath{\textrm{dom($#1$)}}}
\newcommand{\OF}{\mrel{:}}
\newcommand{\mOr}{\mrel{\vee}}
\newcommand{\mAnd}{\mrel{\wedge}}
\newcommand{\mAll}[1]{\ensuremath{\forall} #1.\ms\ms}
\newcommand{\mEx}[1]{\ensuremath{\exists} #1.\ms\ms}
\newcommand{\mExu}[1]{\ensuremath{\exists!} #1.\ms\ms}
\newcommand{\bnfdef}{\mrel{::=}}
\newcommand{\eqdef}{\mrel{:=}}
\newcommand{\set}[1]{\ensuremath{\{#1\}}}

\newcommand{\SysL}{$\lambda$2\xspace}


% Syntactic sorts of the object languages
\newcommand{\TyF}{\ensuremath{\mathsf{Ty_{F}}}}
\newcommand{\TmF}{\ensuremath{\mathsf{Tm_{F}}}}
\newcommand{\TmL}{\ensuremath{\mathsf{Tm_{\lambda}}}}

% Generic type-system judgement predicates
\newcommand{\istyFpr}{\ensuremath{\mathsf{isty_{F}}}}
\newcommand{\typingFpr}{\ensuremath{\mathsf{ofty_{F}}}}
\newcommand{\typingLpr}{\ensuremath{\mathsf{ofty_{\lambda}}}}

% Judgements
\newcommand{\ty}{\mathsf{ty}}
\newcommand{\tm}{\mathsf{tm}}
\newcommand{\of}{\ensuremath{\!:\!}}
\newcommand{\cc}[2]{#1;#2} % compound System F contexts
\makeatletter
\newcommand{\raisemath}[1]{\mathpalette{\raisem@th{#1}}}
\newcommand{\raisem@th}[3]{\raisebox{#1}{\ensuremath{#2#3}}}
\makeatother
\newcommand{\tsAnnot}[2]{\vdash\hspace{-.7em}^{\raisemath{1.5pt}{\scriptscriptstyle{#2}}}_{\raisemath{0.3pt}{\scriptscriptstyle{#1}}}} % note: wrap instances in \mathbin
\newcommand{\cts}[2]{\ensuremath{(\,#1 #2 {})}} % context + turnstile for CMs
\newcommand{\tfF}{\tsAnnot{F}{\ty}}  % for type formation judgement
\newcommand{\tyF}{\tsAnnot{F}{\tm}}  % for typing judgement
\newcommand{\istyF}[2]{\ensuremath{{#1} \mathrel{\tfF} #2}}
\newcommand{\typingF}[3]{\ensuremath{{#1} \mathrel{\tyF} #2 \OF #3}}
\newcommand{\tyL}{\tsAnnot{\lambda}{}} % for typing judgement
\newcommand{\typingL}[3]{\ensuremath{{#1} \mathrel{\tyL} #2 \OF #3}}
\newcommand{\inL}{\mrel{\in_{\lambda}}}
\newcommand{\tfP}{\tsAnnot{\mathsf{P}}{\ty}}  % for type formation judgement
\newcommand{\istyFh}[1]{\ensuremath{#1\ms\mathsf{ty}}}
\newcommand{\typingFh}[2]{\ensuremath{#1 \mathbin{:_{F}} #2}}
\newcommand{\sortLh}[1]{\ensuremath{\mathcal{S}\ms#1}}
\newcommand{\typingLh}[2]{\ensuremath{#1 \mathbin{:_{\lambda}} #2}}


% The type and term relations
\newcommand{\tyr}{\mathrel{\sim}}
\newcommand{\tmr}{\mathrel{\approx}}

\newcommand{\Rext}[1]{\ensuremath{#1^{\mathsf{ext}}}}
\newcommand{\Rshift}[1]{\ensuremath{#1^{\Uparrow}}}

% relational context morphisms
\newcommand{\tyctxrelFL}[3]{\ensuremath{#1\mathrel{\mathop{\longrightarrow}^{#2}\limits}#3}}
\newcommand{\tyctxrelLF}[3]{\ensuremath{#1\mathrel{\mathop{\longleftarrow}^{#2}\limits}#3}}
\newcommand{\tmctxrelFL}[4]{\ensuremath{#1\mathrel{\mathop{\longrightarrow}^{#2}_{#3}\limits}#4}}
\newcommand{\tmctxrelLF}[4]{\ensuremath{#1\mathrel{\mathop{\longleftarrow}^{#2}_{#3}\limits}#4}}

% L-Prolog syntax
\newcommand{\lpPi}[1]{\mathbf{\Pi} #1.\ms\ms}
\newcommand{\lpApp}[2]{#1\langle#2\rangle}
\newcommand{\lpImp}{\mrel{=\!\blacktriangleright}}

% Object Syntax
\newcommand{\Prp}{\ensuremath{\textrm{\textasteriskcentered}}}
\newcommand{\Typ}{\ensuremath{\square}}
\newcommand{\All}{\ensuremath{\forall.\,}}
\newcommand{\nAll}[1]{\ensuremath{\forall #1.\,}}
\newcommand{\Lam}[1]{\ensuremath{\lambda #1.\,}}
\newcommand{\TyLam}{\ensuremath{\Lambda.\,}}
\newcommand{\nTyLam}[1]{\ensuremath{\Lambda #1.\,}}
\newcommand{\Prod}[1]{\ensuremath{\Pi #1.\,}}

\newcommand{\emptyctx}{\ensuremath{\bullet}}

% Substitutions
\newcommand{\subst}[1]{\hphantom{|}\!\![{#1}]}
\newcommand{\scons}{\mathbin{\hspace{0.05em}\cdot\hspace{0.05em}}}
\newcommand{\scomp}{\mathbin{\hspace{-0.1em}{\circ}\hspace{-0.1em}}}
\newcommand{\hscomp}{\mathbin{\hspace{-0.1em}{\hat\circ}\hspace{-0.1em}}}
\newcommand{\id}{\mathsf{id}}
\newcommand{\up}{{\Uparrow}}
%\newcommand{\shift}{\ensuremath{\hspace{0.06em}\textsf{\small +}\!1\hspace{-0.06em}}}
%\newcommand{\ushift}{\ensuremath{\textsf{\small -}\!1}}
\newcommand{\shift}{\ensuremath{\hspace{0.1em}\mathsf{+}\hspace{0.08em}\!1}}
\newcommand{\ushift}{\ensuremath{\hspace{0.1em}\textsf{--}\hspace{0.1em}\!1}}


%%% Content macros::end %%%%%%%%%%%%%%%%%%%%%%%%%%%%%%%%%%%%%%%%%%%%%%%

\begin{document}

\maketitle

\begin{abstract}
  We give three formalisations of a proof of the equivalence of the usual, two-sorted presentation of System~F and its single-sorted pure type system (PTS) variant \SysL.
  This is established by reducing the typability problem of F to \SysL and vice versa.
  A key challenge is the treatment of variable binding and contextual information.
  The formalisations all share the same high level proof structure using relations to connect the type systems.
  They do, however, differ significantly in their representation and manipulation of variables and contextual information.
  In Coq, we use pure de~Bruijn indices and parallel substitutions.
  In Abella, we use higher-order abstract syntax (HOAS) and nominal constants of the ambient reasoning logic.
  In Beluga, we also use HOAS but within contextual modal type theory.
  Our contribution is twofold.
  Firstly we present and compare a collection of machine-checked solutions to a non-trivial theoretical result.
  Secondly we propose our proof as a benchmark, complementing the POPLmark challenge by testing how well a given proof assistant or framework handles complex contextual information involving multiple type systems.
\end{abstract}

\section{Introduction}

TODO

\begin{itemize}
\item Different variants of System~F~\cite{Girard1972, DBLP:conf/programm/Reynolds74} used interchangable, only valid with suitable equivalence results.
  Briefly discussed but not in detail in~\cite{Geuvers1993}
\item Until recently taken for granted, never considered in detail.
\item We have prior work in Coq \cite{KaiserEtAl:2017:sysf_pts_equiv_coq}
\item Now: 3 proofs, all following the same relational structure.
  \begin{itemize}
  \item Coq: Still uses de Bruijn, but inductive relations instead of translation functions.
    Generalises context morphisms to context relations.
  \item Abella: HOAS Syntax and HOAS definition of relations, heavy use of nominal variables at meta-level to represent object-level variables.
    Contexts are unstructured bags of judgements.
    Dangling variables are a possibility.
  \item Beluga: also HOAS encoded, but using contextual modal type theory at the meta level.
    As a consequence contexts have a lot more structure (compared to Abella), and it is further impossible to write down judgements with dangling variables.
    In contrast to Coq and Abella, Beluga does not provide a tactic language so all proofs have to be given as explicit proof terms.
  \end{itemize}
\item Brief History: Abella does not have functions so when we ported our result from Coq to Abella we were forced to use relations instead.
  Since the languages have different expressivity in their ill-typed fragment, it is impossible to establish a full 1-1 correspondence.
  Moreover, even on the well-typed fragments, obtaining the correspondence is only relative to a suitable correspondence of free variables.
  Due to these considerations of partiality it turns out that relations are in fact the more natural choice to formulate and prove the problem.
  Since Abella contexts don't exhibit a lot of structure out of the box, this structure has to be retrofitted with suitable predicates over contexts, together with corresponding lookup and inversion principles.
  This then brings us to Beluga, where the underlying theory provides the requisite structures and thus removes a lot of boilerplate.
  To demonstrate that the use of relations instead of functions is independent of the concrete syntax representation, we back-ported the Abella/Beluga proof to Coq and where happy to find that the use of de Bruijn syntax does not lead to any significant obstacles.
  Doing the proof this many times has clearly delineated which aspects are inherent to the problem and which are simply artefacts of the machinery in use.
\end{itemize}

Suggestions and Observations:
\begin{itemize}
\item HOAS should really be understood as a high level abstraction layer, then either a system supports it directly, or it has an underlying implementation of this abstraction, e.g.\ via de Bruijn with parallel substitutions (see HYBRID for Coq).
\item Our Challenge stands orthogonal to the POPLmark Challenge~\cite{poplmark}
\end{itemize}

\section{Equivalence}
\label{sec:equivalence}

\begin{figure}
  \begin{center}
    \begin{align*}
      A, B &\bnfdef X \mid A \to B \mid \nAll X A & s, t &\bnfdef x \mid s\,t \mid \Lam {x \of A} s \mid s\,A \mid \nTyLam X s
    \end{align*}
    \begin{mathpar}
      \inferrule*{X \in \Delta}{\istyF{\Delta}{X}} \and
      \inferrule*{\istyF{\Delta}{A} \\ \istyF{\Delta}{B}}{\istyF{\Delta}{A \to B}} \and
      \inferrule*[right=$X \notin \Delta$]{\istyF{\Delta,X}{A}}{\istyF{\Delta}{\nAll X A}} \and
      \inferrule*{\Gamma(x)=A \\ \istyF{\Delta}{A}}{\typingF{\cc{\Delta}{\Gamma}}{x}{A}} \\
      \inferrule*{\typingF{\cc{\Delta}{\Gamma}}{s}{A \to B} \\ \typingF{\cc{\Delta}{\Gamma}}{t}{A}}{\typingF{\cc{\Delta}{\Gamma}}{s\,t}{B}} \and
      \inferrule*[right=$x \notin \dom{\Gamma}$]{\typingF{\cc{\Delta}{\Gamma,x \of A}}{s}{B} \\ \istyF{\Delta}{A}}{\typingF{\cc{\Delta}{\Gamma}}{\Lam {x \of A} s}{A \to B}} \\
      \inferrule*{\typingF{\cc{\Delta}{\Gamma}}{s}{\nAll X B} \\ \istyF{\Delta}{A}}{\typingF{\cc{\Delta}{\Gamma}}{s\, A}{B\subst{A/X}}} \and
      \inferrule*[right=$X \notin \Delta$]{\typingF{\cc{\Delta, X}{\Gamma}}{s}{A}}{\typingF{\cc{\Delta}{\Gamma}}{\nTyLam X s}{\nAll X A}}
    \end{mathpar}
  \end{center}
  \caption{System~F, two-sorted with explicit type variable context $\Delta$.}
  \label{fig:sys-f}
\end{figure}

\begin{figure}
  \begin{center}
    \begin{align*}
      a, b &\bnfdef s \mid x \mid a\,b \mid \Lam a b \mid \Prod a b & s &\in \set{\Prp, \Typ}
    \end{align*}
    \begin{mathpar}
      \inferrule*{~}{\typingL{\Gamma}{\Prp}{\Typ}} \and
      \inferrule*{x \of a \in \Gamma \\ \typingL{\Gamma}{a}{s}}{\typingL{\Gamma}{x}{a}} \and
      \inferrule*[right=$x \notin \dom{\Gamma}$]{\typingL{\Gamma}{a}{s} \\ \typingL{\Gamma,x \of a}{b}{\Prp}}{\typingL{\Gamma}{\Prod{x \of a} b}{\Prp}} \\
      \inferrule*{\typingL{\Gamma}{a}{\Prod{x \of c} d} \\ \typingL{\Gamma}{b}{c}}{\typingL{\Gamma}{a\, b}{d\subst{b/x}}} \and
      \inferrule*[right=$x \notin \dom{\Gamma}$]{\typingL{\Gamma}{a}{s} \\ \typingL{\Gamma, x \of a}{b}{c} \\ \typingL{\Gamma, x \of a}{c}{\Prp}}{\typingL{\Gamma}{\Lam{x \of a} b}{\Prod{x \of a} c}}
    \end{mathpar}
  \end{center}
  \caption{\SysL, single-sorted, uniform pure type system with dependent context.}
  \label{fig:sys-l}
\end{figure}

We consider two variants of System F.

The version in Figure~\ref{fig:sys-f} is a standard, two-sorted presentation that cleanly separates terms and types.
We refer to this variant in the following simply as F.
For a number of reasons we chose to include an explicit type variable context and a type formation judgement, similar to the one found in \cite{Harper2013}.
The main reason for this that, depending on the representation of choice, the well-formedness of types may be immediate (as is the case in Beluga) or left as a proof obligation (as in Abella and Coq).
Moreover, explicitly tracking this information brings the system closer to our PTS, which inherently tracks type formation information.

The second variant is the uniform, single-sorted PTS \SysL given in Figure~\ref{fig:sys-l}.
This system is very close to the respective corner in Barendregt's $\lambda$-cube~\cite{DBLP:journals/jfp/Barendregt91}, though we omit the conversion rule.
This is justified since well-formed \SysL types do not contain any $\beta$-redices.

Note that neither of the systems requires their contexts to be completely well-formed.
To ensure that the type systems are sill meaningful we check the well-formedness of types, whenever we add them to or extract them from the context.
So effectively we require contexts to be semi-well-formed, that is well-formed on the types that are actually accessed.
This design decision has a number of interesting consequences, as we will see later.
It is of course possible to restrict the proof to well-formed contexts (and in Beluga this happens almost automatically), but it can be helpful, and indeed in \cite{KaiserEtAl:2017:sysf_pts_equiv_coq} it was essential, to leave the constraint off.

Where necessary, well-formedness of F-contexts can be defined as follows:
\begin{mathpar}
  \inferrule*{~}{\mathbf{wf}_{F}\;\;\cc{\emptyset}{\emptyctx}} \and
  \inferrule*[right=$X \notin \Delta$]{\mathbf{wf}_{F}\;\;\cc{\Delta}{\emptyctx}}{\mathbf{wf}_{F}\;\;\cc{\Delta,X}{\emptyctx}} \and
  \inferrule*[right=$x \notin \dom{\Gamma}$]{\mathbf{wf}_{F}\;\;\cc{\Delta}{\Gamma} \\ \istyF{\Delta}{A}}{\mathbf{wf}_{F}\;\;\cc{\Delta}{\Gamma, x \of A}}
\end{mathpar}

Note that the second rule is restricted to the empty term-variable context.
This is to align the well-formedness relation with an internalisation function
\begin{align*}
  \mathsf{intern}_{F} \OF \mathsf{CtxTy}_{F} \to \mathsf{CtxTm}_{F} \to \TyF \to \TmF \to \TmF \times \TyF
\end{align*}
which satisfies
\begin{align*}
  \mathbf{wf}_{F}\;\;\cc{\Delta}{\Gamma} \implies \mathsf{intern}_{F}\,\Delta\,\Gamma\,s\,A = (t,B) \implies (\typingF{\cc{\Delta}{\Gamma}}{s}{A} \iff \typingF{\cc{\emptyset}{\emptyctx}}{t}{B})
\end{align*}
This is what allows us to only consider closed judgements at the top-level.
The non-empty contexts of open judgements can always be internalised.
For \SysL we argue similarly.

The objective of the benchmark is to formally establish the equivalence of these two languages.
More precisely, the goal is a bidirectional reduction of the typability problem.

In \cite{KaiserEtAl:2017:sysf_pts_equiv_coq} such an equivalence result was established via syntactic translation functions.
The reduction result then takes the following form:
\begin{align*}
  A &\iff B\\
  C &\iff D
\end{align*}
Due to various alignment mismatches between the two languages, one direction of the above translation is necessarily partial.
This partiality leads to quite a number of technically intricate complications throughout the proof.

At the heart of the problem sits the fact that only the well-typed fragments of the languages should be considered as relevant, the rest is ``junk''.
To obtain the reduction results using syntactic translation functions it is however necessary to handle this ``junk'' and face the fact that outside of the well-typed fragment there are significant differences in expressivity.
So in order to avoid these issues we opt for a relational approach that can concentrate exclusively on the meaningful parts of the languages.

The idea is to construct two relations $\tyr$ and $\tmr$ that put the types and terms of the two languages in correspondence.
The equivalence statement than factors into a number of properties that these relations should exhibit.

\subsection{The Benchmark / Challenge}
\label{sec:benchmark--challenge}

\begin{enumerate}
\item Define the syntactic sorts $\TyF$, $\TmF$ and $\TmL$ to encode the languages of Figures~\ref{fig:sys-f}~and~\ref{fig:sys-l}, respectively.
\item For each language, define the corresponding type system.
  That is, define predicates
  \begin{align*}
    \istyFpr &\OF \TyF \to \Prop\\
    \typingFpr &\OF \TmF \to \TyF \to \Prop\\
    \typingLpr &\OF \TmL \to \TmL \to \Prop
  \end{align*}
  When the encodings of the languages are explicitly tracking contextual information, then further arguments may be required.
\item Define two relations that associate the types and, respectively, the terms of the two languages:
  \begin{align*}
    \tyr &\OF \TyF \to \TmL \to \Prop\\
    \tmr &\OF \TmF \to \TmL \to \Prop
  \end{align*}
  It may again be necessary to augment these with additional contextual information.
  We will use infix notation, e.g.\ $A \tyr a$.
\item Prove the following properties:
  \begin{itemize}
  \item $\tyr$ is functional and injective.
  \item $\tyr$ is left-total and type-formation preserving on the well-formed types of F.
  \item $\tyr$ is right-total and type-formation preserving on the propositions of \SysL.
  \item $\tmr$ is functional and injective.
  \item $\tmr$ is left-total and typing preserving on the well-typed terms of F.
  \item $\tmr$ is right-total and typing preserving on the proofs of \SysL.
  \end{itemize}
  Note that a proposition of \SysL is any term $a \OF \TmL$ such that $\typingLpr\,a\,\Prp$ holds.
  A proof of \SysL is any term $b \OF \TmL$ such that $\typingLpr\,b\,a$ holds for $a$ a proposition of \SysL.
\end{enumerate}
At this point we can formulate, and easily prove, the following equivalences:

\begin{theorem}[Reductions from F to \SysL]
  \begin{align*}
    \istyFpr\,A &\iff \mExu{a} A \tyr a \mAnd \typingLpr\,a\,\Prp\\
    \typingFpr\,s\,A &\iff \mExu{b a} s \tmr b \mAnd A \tyr a \mAnd \typingLpr\,b\,a \mAnd \typingLpr\,a\,\Prp
  \end{align*}
\end{theorem}

\begin{proof}
  The forward directions are simply the corresponding left-to-right preservation and left-totality results of $\tmr$ and $\tyr$.
  Uniqueness follows from functionality.
  For the inverse direction we use preservation (here from right to left) and uniqueness.
\end{proof}

\begin{theorem}[Reductions from \SysL to F]
  \begin{align*}
    \typingLpr\,a\,\Prp &\iff \mExu{A} A \tyr a \mAnd \istyFpr\,A\\
    \typingLpr\,b\,a \mAnd \typingLpr\,a\,\Prp &\iff \mExu{s A} s \tmr b \mAnd A \tyr a \mAnd \typingFpr\,s\,A
  \end{align*}
\end{theorem}

\begin{proof}
  Dual to the previous result.
\end{proof}

\section{Coq}
\label{sec:coq}

Let us next consider the first concrete instance of our benchmark proof, namely the variant in Coq that utilises pure de Bruijn syntax with parallel substitutions and inductively defined type and term relations.

We begin with a brief survey of our syntax.
Binders are not annotated with variable names and variables themselves are simply numerical indices, where index $n$ references the $n$th enclosing binder.
Dangling indices count into the corresponding context.
We use the Autosubst framework~\cite{DBLP:conf/itp/SchaferTS15}, to automatically generate substitutions and a normalisation procedure which allows to decide the equality of two terms (or types) with applied substitutions.
Substitutions are parallel, that is, they are functions from indices to terms (or types) which act simultaneously on all free indices of a term (or type).

\begin{figure}
  \begin{center}
    \begin{align*}
      A, B &\bnfdef x_\ty \mid A \to B \mid \All A & s, t &\bnfdef x_\tm \mid s\,t \mid \Lam A s \mid s\,A \mid \TyLam s & &x \OF \Nat
    \end{align*}
    \begin{mathpar}
      \inferrule*{x < N}{\istyF{N}{x_\ty}} \and
      \inferrule*{\istyF{N}{A} \\ \istyF{N}{B}}{\istyF{N}{A \to B}} \and
      \inferrule*{\istyF{N+1}{A}}{\istyF{N}{\All A}} \and
      \inferrule*{A_x = A \\ \istyF{N}{A}}{\typingF{\cc{N}{A_n,\ldots, A_0}}{x_\tm}{A}} \\
      \inferrule*{\typingF{\cc{N}{\Gamma}}{s}{A \to B} \\ \typingF{\cc{N}{\Gamma}}{t}{A}}{\typingF{\cc{N}{\Gamma}}{s\,t}{B}} \and
      \inferrule*{\typingF{\cc{N}{\Gamma,A}}{s}{B} \\ \istyF{N}{A}}{\typingF{\cc{N}{\Gamma}}{\Lam A s}{A \to B}} \\
      \inferrule*{\typingF{\cc{N}{\Gamma}}{s}{\All A} \\\istyF{N}{B}}{\typingF{\cc{N}{\Gamma}}{s\, B}{A\subst{B\scons\id}}} \and
      \inferrule*{\typingF{\cc{N+1}{\Gamma\subst{\shift}}}{s}{A}}{\typingF{\cc{N}{\Gamma}}{\TyLam s}{\All A}}
    \end{mathpar}
  \end{center}
  \caption{System~F -- de Bruijn encoding in Coq.}
  \label{fig:sys-f-coq}
\end{figure}

\begin{figure}%[t] s.o.
  \begin{center}
    \begin{align*}
      s &\bnfdef \Prp \mid \Typ & a, b, c, d &\bnfdef s \mid x \mid a\,b \mid \Lam a b \mid \Prod a b & &x \OF \Nat
    \end{align*}
    \begin{mathpar}
      \inferrule*{~}{0 \of a\subst{\shift} \inL \Gamma,a}\and
      \inferrule*{x \of a \inL \Gamma}{(x+1) \of a\subst{\shift} \inL \Gamma,b}\and
      \inferrule*{~}{\typingL{\Gamma}{\Prp}{\Typ}} \and
      \inferrule*{x \of a \inL \Gamma \\ \typingL{\Gamma}{a}{s}}{\typingL{\Gamma}{x}{a}} \\
      \inferrule*{\typingL{\Gamma}{a}{s} \\ \typingL{\Gamma,a}{b}{\Prp}}{\typingL{\Gamma}{\Prod a b}{\Prp}} \and
      \inferrule*{\typingL{\Gamma}{a}{\Prod c d} \\ \typingL{\Gamma}{b}{c}}{\typingL{\Gamma}{a\, b}{d\subst{b\scons\id}}} \and
      \inferrule*{\typingL{\Gamma}{a}{s} \\\\ \typingL{\Gamma, a}{b}{c} \\ \typingL{\Gamma, a}{c}{\Prp}}{\typingL{\Gamma}{\Lam a b}{\Prod a c}}
    \end{mathpar}
  \end{center}
  \caption{\SysL -- de Bruijn encoding in Coq.}
  \label{fig:sys-l-coq}
\end{figure}

\begin{itemize}
\item Figure~\ref{fig:sys-f-coq} gives the de Bruijn encoding of F.
  Figure~\ref{fig:sys-l-coq} gives the de Bruijn encoding of \SysL.
  For the latter observe the dependent context lookup necessary due to de Bruijn term not being invariant under context modifications.
\item Remark on detour via System~P to obtain strengthening for type formation and propagation in \SysL without assuming well-formedness.
\end{itemize}

Now that we have a good understanding of how our systems operate concretely, let us consider how we relate them.
What should be immediately apparent is that since our systems are explicitly tracking context information in the typing judgements, something similar will have to occur for the type and term relations.
More precisely, before we can, for example, ascertain which types are related, we have to be able to track which type variables are related.
If this sounds reminiscent of the discussion above that led to the notion of context morphism lemmas then the reader is certainly on the right track.

We start with relations on indices, $R \subseteq \Nat \times \Nat$.
Since we are going to transform such relations in a way that mirrors how parallel substitutions change when they traverse binders we are now looking at our concrete implementation, that is we define a type of variable relations $\mathsf{vr} \eqdef \mathsf{list}\,(\mathsf{var} \times \mathsf{var})$.
Next we consider a function
\begin{align*}
  \mathsf{bimap} \OF (\mathsf{var} \to \mathsf{var}) \to (\mathsf{var} \to \mathsf{var}) \to \mathsf{vr} \to \mathsf{vr}
\end{align*}
which satisfies
\begin{align*}
  &x\,R\,y \to (\xi\,x)\,(\mathsf{bimap}\,\xi\,\zeta\,R)\,(\zeta\,y)\\
  &x\,(\mathsf{bimap}\,\xi\,\zeta\,R)\,y \to \mEx{x'\,y'} x'\,R\,y' \wedge x = \xi\,x' \wedge y = \zeta\,y'
\end{align*}
We are going to require two particular operations on such variable relations, which both build on top of $\mathsf{bimap}$.
The first encapsulates the idea of moving, in lockstep, underneath a binder on both sides the term (or type) relation we are in the process of setting up.
In order to achieve this, we have to make sure that the indices $0$ on the left and $0$ on the right are related, which ties the two binders that were just traversed, together.
Furthermore, all indices that are already related have to be shifted up by one.
We obtain:
\begin{align*}
  \Rext{R} \eqdef (0,0) \mathop{::} \mathsf{bimap}\,(\shift)\,(\shift)\,R
\end{align*}
The other operation deals with the fact that we sometimes traverse a binder on one side, that has no counterpart on the other.
As a consequence, only one component of each tuple needs to be shifted. We define
\begin{align*}
  \Rshift{R} \eqdef \mathsf{bimap}\,\id\,(\shift)\,R
\end{align*}
Note that we do not require the dual shift on the left, since we always have the PTS on the right of our relations.

\begin{lemma}
  Both $\Rext{R}$ and $\Rshift{R}$ preserve injectivity and functionality of $R$.
\end{lemma}

\begin{proof}
  Straightforward, using the properties of $\mathsf{bimap}$.
\end{proof}

We now have all the ingredients to inductively define our relations $\tyr$ and $\tmr$:
\begin{mathpar}
  \inferrule*{x\,R\,y}{x_\ty\,\tyr^R\,y} \and
  \inferrule*{A\,\tyr^R\,a \\ B\,\tyr^{\Rshift{R}}\,b}{A \to B\,\tyr^R\,\Prod a b} \and
  \inferrule*{A\,\tyr^{\Rext{R}}\,a}{\All A\,\tyr^R\,\Prod \Prp a}\\
  \inferrule*{x\,S\,y}{x_\tm\,\tmr^R_S\,y} \and
  \inferrule*{s\,\tmr^R_S\,a \\ t\,\tmr^R_S\,b}{s\,t\,\tmr^R_S\,a\,b} \and
  \inferrule*{s\,\tmr^R_S\,a \\ A\,\tyr^R\,b}{s\,A\,\tmr^R_S\,a\,b}\\
  \inferrule*{A\,\tyr^R\,a \\ s\,\tmr^{\Rshift{R}}_{\Rext{S}}\,b}{\Lam A s\,\tmr^R_S\,\Lam a b} \and
  \inferrule*{s\,\tmr^{\Rext{R}}_{\Rshift{S}}\,a}{\TyLam s\,\tmr^R_S\,\Lam \Prp a}
\end{mathpar}

\begin{lemma}
  The type relation $\tyr^R$ is injective/functional, whenever $R$ is injective/functional.
\end{lemma}
\begin{proof}
  Straightforward inductions.
\end{proof}
Proving the left and right totality and preservation results is slightly more interesting, as we have to generalise to open judgements and non-empty contexts.
We start with the direction from left to right and define

\begin{align*}
  \tyctxrelFL{N}{R}{\Gamma} \eqdef \mAll {x < N} \mEx y x\,R\,y \mAnd y \of \Prp \inL \Gamma
\end{align*}
which satisfies the following:
\begin{align*}
  &\tyctxrelFL{0}{R}{\Gamma} & &\tyctxrelFL{N}{R}{\Gamma} \implies \tyctxrelFL{N}{\Rshift{R}}{\Gamma,a} & &\tyctxrelFL{N}{R}{\Gamma} \implies \tyctxrelFL{N+1}{\Rext{R}}{\Gamma,\Prp}
\end{align*}

\begin{lemma}
  \label{lem:tyr_fl_tot_pres}
  The type-relation $\tyr^R$ is total from left to right and preserves type formation:
  \begin{align*}
    \istyF{N}{A} \implies \mAll {R\,\Gamma} \tyctxrelFL{N}{R}{\Gamma} \implies \mEx a A\,\tyr^R\,a \mAnd \typingL{\Gamma}{a}{\Prp}
  \end{align*}
\end{lemma}
\begin{proof}
  By induction on $\istyF{N}{A}$. The two binder cases are handled via the respective properties of $\tyctxrelFL{N}{R}{\Gamma}$.
\end{proof}
Let us briefly pause at this point and consider what we just did.
If we compare the previous result to what we did in \cite{KaiserEtAl:2017:sysf_pts_equiv_coq} it will quickly become apparent that $\tyctxrelFL{N}{R}{\Gamma}$ is the relational generalisation of the context morphism $\xi\ms:\ms\cts{N}{\tfF}\to\cts{\Gamma}{\tfP}$.
Thus Lemma~\ref{lem:tyr_fl_tot_pres} is simply the relational analogue to the context morphism lemma for type formation from F to P (Lemma~15) in \cite{KaiserEtAl:2017:sysf_pts_equiv_coq}.
As noted above, our relational approach allows us to bypass the intermediate P from the earlier paper, that is we relate F and \SysL directly.

For the direction from right to left we simply give the respective relational morphism definition and state the result:
\begin{lemma}
  \label{lem:tyr_lf_tot_pres}
  The type-relation $\tyr^R$ is total from right to left and preserves type formation.
  Let $\tyctxrelLF{N}{R}{\Gamma} \eqdef \mAll {y} y\of\Prp \inL \Gamma \implies \mEx x x\,R\,y \mAnd x < N$, then
  \begin{align*}
    \typingL{\Gamma}{a}{\Prp} \implies \mAll {R\,N} \tyctxrelLF{N}{R}{\Gamma} \implies \mEx A A\,\tyr^R\,a \mAnd \istyF{N}{A}
  \end{align*}
\end{lemma}
\begin{proof}
  Induction on $\typingL{\Gamma}{a}{\Prp}$, using extension properties for $\tyctxrelLF{N}{R}{\Gamma}$.
  The correct function space is chosen by discriminating on the sort of the product domain.
  This in turn relies on the degeneracy of the sort $\Typ$.
  Finally, propagation and substitutivity of $\tyL$ are used to discharge the spurious application case, which has no related type on the F-side
\end{proof}

Before we can now tackle the term relation $\tmr^R_S$ we need a bit of extra machinery.
First of all we require $\beta$-substitutivity for $\tyr^R$, that is
\begin{align*}
  B\,\tyr^R\,b \implies A\,\tyr^{\Rext{R}}\,a \implies A\subst{B\scons\id}\,\tyr^R\,a\subst{b\scons\id}
\end{align*}
The solution to this is again a CML-style generalisation, namely
\begin{align*}
  A\,\tyr^{R_1}\,a \implies \mAll {\sigma\,\tau\,R_2} R_1\stackrel{\sigma\mid\tau}{\longrightarrow}R_2 \implies A\subst{\sigma} \tyr^{R_2}\,a\subst{\tau}.
\end{align*}
where the definition of $R_1\stackrel{\sigma\mid\tau}{\longrightarrow}R_2$ should be obvious.
Note that we have to have $\sigma$ and $\tau$ as full de Bruijn substitutions, which prevents us from simply using $\mathsf{bimap}$ here.


We further recall that $\tmr^R_S$ actually uses two variable relations, $R$ for the type variables and $S$ for the term variables.
In the following it is essential that these two relations are -- and remain -- disjoint with respect to their image, which we denote as $R \| S$.
We observe the following facts:
\begin{align*}
  R \| S &\implies \Rshift{R} \| \Rext{S} & R \| S &\implies \Rext{R} \| \Rshift{S}
\end{align*}
We can then lift this easily to our type and term relations, namely
\begin{align*}
  R \| S \implies A\,\tyr^R\,a \implies s\,\tmr^R_S\,a \implies \bot
\end{align*}

\begin{lemma}
  When $R$ and $S$ are functional then so is $\tmr^R_S$.
  When $R$ and $S$ are injective and $R \| S$ holds, then $\tmr^R_S$ is injective.
\end{lemma}

Now all that is left are the proofs of left and right totality as well as preservation of typing for $\tmr^R_S$.
This, however, is the most technical part of the development.
There are no additional major ideas involved in the following, but the number of moving parts increases noticeably.
We start with the direction from F to \SysL and define
\begin{align*}
  \tmctxrelFL{\Delta}{R}{S}{\Gamma} \eqdef \mAll {x \of A \in \Delta} \mEx{y\,a} A\,\tyr^R\,a \mAnd x\,S\,y \mAnd y \of a \inL \Gamma
\end{align*}
which satisfies
\begin{align*}
  &\tmctxrelFL{\Delta}{R}{S}{\Gamma} \implies A\,\tyr^R\,a \implies \tmctxrelFL{\Delta,A}{\Rshift{R}}{\Rext{S}}{\Gamma,a}\\
  &\tmctxrelFL{\Delta}{R}{S}{\Gamma} \implies \tmctxrelFL{\Delta\subst{\shift}}{\Rext{R}}{\Rshift{S}}{\Gamma,\Prp}
\end{align*}
\begin{lemma}
  The term relation $\tmr^R_S$ is total from left to right and preserves typing.
  \begin{align*}
    \typingF{\cc{N}{\Delta}}{s}{A} \implies \mAll{R\,S\,\Gamma} &R\ms\ms\mathsf{func} \implies \tyctxrelFL{N}{R}{\Gamma} \implies \tmctxrelFL{\Delta}{R}{S}{\Gamma} \implies \\
                                                                &\mEx{a\,b} A\,\tyr^R\,b \mAnd s\,\tmr^R_S\,a \mAnd \typingL{\Gamma}{a}{b} \mAnd \typingL{\Gamma}{b}{\Prp}
  \end{align*}
\end{lemma}
\begin{proof}
  By induction on $\typingF{\cc{N}{\Delta}}{s}{A}$.
  Both the invariant $\tmctxrelFL{\Delta}{R}{S}{\Gamma}$, as well as Lemma~\ref{lem:tyr_fl_tot_pres}, are used to obtain related types in the variable case.
  Functionality allows us to equate these.
  For term-level application we require the strengthening for sort-typings in \SysL mentioned above: $\typingL{\Gamma,a}{b\subst{\shift}}{s} \implies \typingL{\Gamma}{b}{s}$.
  Meanwhile the type-level application case requires $\beta$-substitutivity of both $\tyL$ and $\tyr^R$.
  Both abstraction cases are relatively straightforward, given suitable extension laws.
\end{proof}

For the inverse direction we define
\begin{align*}
  \tmctxrelLF{\Delta}{R}{S}{\Gamma} \eqdef \mAll{y\,a} y \of a \inL \Gamma \implies \typingL{\Gamma}{a}{\Prp} \implies \mEx{x\,A} A\,\tyr^R\,a \mAnd x\,S\,y \mAnd x \of A \in \Delta
\end{align*}
which satisfies
\begin{align*}
  &\tmctxrelLF{\Delta}{R}{S}{\Gamma} \implies A\,\tyr^R\,a \implies \tmctxrelLF{\Delta,A}{\Rshift{R}}{\Rext{S}}{\Gamma,a}\\
  &\tmctxrelLF{\Delta}{R}{S}{\Gamma} \implies \tmctxrelLF{\Delta\subst{\shift}}{\Rext{R}}{\Rshift{S}}{\Gamma,\Prp}
\end{align*}
\begin{lemma}
  The term relation $\tmr^R_S$ is total from right to left and preserves typing:
  \begin{align*}
    \typingL{\Gamma}{b}{\Prp} \implies \typingL{\Gamma}{a}{b} \implies \mAll{R\,S\,N\,\Delta} &R\ms\ms\mathsf{inj} \implies \tyctxrelLF{N}{R}{\Gamma} \implies \tmctxrelLF{\Delta}{R}{S}{\Gamma} \implies \\
                                                                &\mEx{s\,A} A\,\tyr^R\,b \mAnd s\,\tmr^R_S\,a \mAnd \typingF{\cc{N}{\Delta}}{s}{A} \mAnd \istyF{N}{A}
  \end{align*}
\end{lemma}
\begin{proof}
  By induction on $\typingL{\Gamma}{a}{b}$.
  The cases are mostly analogue to the previous result.
  Injectivity of $R$ is required for the variable case.
  Note that we need to discriminate on the sorts of product domains to disambiguate the unified abstractions and applications correctly.
\end{proof}

In summary, the most interesting part here was the handling of the contextual information about corresponding free de Bruijn indices.
In the present formalisation this was achieved with relations on indices, which were constrained in such a way that they connect corresponding contexts.
On top of this, modifications were defined that mirror the adjustments of parallel substitutions during the traversal of binders.
Our design of this structure was heavily influenced by the generalised notions of context morphism laid out in~\cite{KaiserEtAl:2017:sysf_pts_equiv_coq}.

\section{Abella}
\label{sec:abella}

\begin{figure}
  \begin{center}
    \begin{mathpar}
%      \mprset{fraction={{}{\,\cdot\,}{}}}
      \inferrule*{\istyFh{A} \\ \istyFh{B}}{\istyFh{(A \to B)}} \and
      \inferrule*{\lpPi x \istyFh{x} \lpImp \istyFh{\lpApp{A}{x}}}{\istyFh{(\All{A})}} \and
      \inferrule*{\typingFh{s}{\All{B}} \\ \istyFh{A}}{\typingFh{s\,A}{\lpApp{B}{A}}} \\
      \inferrule*{\lpPi x \istyFh{x} \lpImp \typingFh{\lpApp{s}{x}}{\lpApp{A}{x}}}{\typingFh{\TyLam{s}}{\All{A}}} \and
      \inferrule*{\istyFh{A} \\ \lpPi x \typingFh{x}{A} \lpImp \typingFh{\lpApp{s}{x}}{B}}{\typingFh{\Lam A s}{A \to B}} \and
      \inferrule*{\typingFh{s}{A \to B} \\ \typingFh{t}{A}}{\typingFh{s\,t}{B}}
    \end{mathpar}
  \end{center}
  \caption{HOAS specification of F in Abella.}
  \label{fig:sys-f-abella}
\end{figure}

\begin{figure}
  \begin{center}
    \begin{mathpar}
%      \mprset{fraction={{}{\,\cdot\,}{}}}
      \inferrule*{~}{\sortLh{\Typ}} \and
      \inferrule*{~}{\sortLh{\Prp}} \and
      \inferrule*{~}{\typingLh{\Prp}{\Typ}} \and
      \inferrule*{\typingLh{a}{\Prod{c}{d}} \\ \typingLh{b}{c}}{\typingLh{a\,b}{\lpApp{d}{b}}} \\
      \inferrule*{\typingLh{a}{s} \\ \sortLh{s} \\\\ \lpPi x \typingLh{x}{a} \lpImp \typingLh{\lpApp{b}{x}}{\Prp}}{\typingLh{\Prod{a}{b}}{\Prp}} \and
      \inferrule*{\typingLh{a}{s} \\ \sortLh{s} \\\\ \lpPi x \typingLh{x}{a} \lpImp \typingLh{\lpApp{c}{x}}{\Prp} \\ \lpPi x \typingLh{x}{a} \lpImp \typingLh{\lpApp{b}{x}}{\lpApp{c}{x}}}{\typingLh{\Lam{a}{b}}{\Prod{a}{c}}}
    \end{mathpar}
  \end{center}
  \caption{HOAS specification of \SysL in Abella.}
  \label{fig:sys-l-abella}
\end{figure}

Let us now consider the first of our two HOAS solutions in more detail.
Abella is a system designed around a two-level logic approach.
The lower \emph{specification level}, essentially verbatim $\lambda$Prolog, is used to encode the object languages.
For our Abella variant of F we define predicates $\istyFh{A} \OF o$ for type formation and $\typingFh{s}{A} \OF o$ for typing, where $o$ is Abella's type of $\lambda$Prolog predicates.
The Abella variant of the type system of F is given in Figure~\ref{fig:sys-f-abella}.
The binders here are again nameless, but this time due to the fact that object level binding is delegated to the binding of the specification logic.
That is in $\Lam{A}{s}$ the body $s \OF \TmF \to \TmF$ is a specification logic function.
We use $\lpApp{s}{t}$ denote the application of specification logic function $s$ to an argument $t$.

To encode \SysL we define a unary predicate $\sortLh{s} \OF o$ to recognise the sorts $s$ and the typing predicate $\typingLh{a}{b} \OF o$.
The definition of the type system is given in Figure~\ref{fig:sys-l-abella}.

In Abella, reasoning about this specifications happens at the \emph{meta level}, using the logic $\mathcal{G}$.
This reasoning logic is the intuitionistic predicative fragment for Church's simple type theory, extended with natural induction, (co)inductive predicates and, for our exposition most relevant, nominal quantification ($\nabla x . \ms s$) -- a form of universal quantification, where the bound identifier is guaranteed to be suitably fresh.

The two levels are connected as follows: whenever a $\lambda$Prolog judgement $J \OF o$ holds than it has a $\lambda$Prolog derivation, and such derivations exhibit an inductive structure.
This inductive structure is accessible in $\mathcal{G}$ with propositions of the form $\{J\}$.
More generally, the derivation of $J$ may depend on hypotheses $I_0,\ldots,I_n$.
This set of hypotheses is exposed in the form of a list $L \OF \mathsf{list}\,o$, and the embedding is written as $\{L \vdash J\}$.
At this point it is interesting to observe what happens with hypothetical ($\lpImp$) and $\mathbf{\Pi}$-quantified premises.
Consider the premise of F type formation rule for $\All A$.
Embedded in $\mathcal{G}$ it takes the form $\{L \vdash\ \lpPi x \istyFh{x} \lpImp \istyFh{\lpApp{A}{x}}\}$ which is immediately converted to $\nabla x.\ms \{L, \istyFh{x} \vdash \istyFh{\lpApp{A}{x}}\}$.
The crucial thing to note is that $L$ and $A$ are both quantified outside of the $\nabla$, hence $\mathcal{G}$ enforces that $x$ is fresh for these two quantities.
In other words $x$ really is a variable in the usual sense.
Abella usually goes a step further and opens the quantification with a fresh nominal constant $n_i$ that is syntactically distinct from other quantities in the proof state.
Let us briefly stay with this example and consider the following goal, where $n_1$ is a nominal constant:
\begin{mathpar}
  \mprset{flushleft,
    fraction={===},
    fractionaboveskip=0.5ex,
    fractionbelowskip=0.5ex}
  \inferrule*{\{L \vdash \istyFh{B}\} \\\\ \{L, \istyFh{n_1} \vdash \istyFh{\lpApp{A}{n_1}}\}}{\{L \vdash \istyFh{\lpApp{A}{B}}\}}
\end{mathpar}
The embedding of $\lambda$Prolog in $\mathcal{G}$ is equipped with generic instantiation and cut theorems that are made available to the user via specific tactics.
With these we can instantiate the second premise above to $\{L, \istyFh{B} \vdash \istyFh{\lpApp{A}{B}}\}$.
Then cutting this with the first premise yields the goal.
This generic cut tactic immediately provides substitutivity results for object languages, something that usually takes a significant amount of work in low-level representations like de Bruijn.

There is one aspect of the embedding that is prone to cause headaches, namely the treatment of contexts.
Contexts are made available as lists of judgements of type $o$, but there is nothing that a priori guarantees that some $I_k \in L$ actually is of the form $\istyFh A$ for some $A$.
Moreover, since these context should represent typing contexts (or type formation contexts) they should only really contain assumptions about variables.
And variables of the object language appear in $\mathcal{G}$ as nominal constants.
So the $L$ in $\{L \vdash \istyFh{A}\}$ should be of the form
\begin{align*}
  \istyFh{n_i}, \istyFh{n_j}, \istyFh{n_k}, \ldots
\end{align*}
Meanwhile, for the derivation of a typing judgement $\{L \vdash \typingFh{s}{A}\}$, the elements of $L$ should only of the form $\istyFh{n_i}$ or $\typingFh{n_k}{A}$ (where $A$ is well-formed).
We can achieve this with an inductive $\mathcal{G}$-predicate that ensures the well-formedness of a F-typing context:
\newcommand{\ac}[3]{\ensuremath{\mathbb{C}_{#1}^{#2}\ms (#3)}}
\newcommand{\acFty}[1]{\ac{F}{\ty}{#1}}
\newcommand{\acFtm}[1]{\ac{F}{\tm}{#1}}
\begin{mathpar}
  \inferrule*{~}{\acFtm{\emptyctx}} \and
  \inferrule*{\acFtm{L} \\ x \notin L}{\acFtm{L,\istyFh{x}}}\and
  \inferrule*{\acFtm{L} \\ \{L \vdash \istyFh{A}\} \\ x \notin L,A}{\acFtm{L,\typingFh{x}{A}}}
\end{mathpar}
In Abella, the freshness assumptions are enforced through nominal quantification:
\begin{align*}
  &\mathbf{Define}\ms\ms\mathsf{ctxftm} \OF \mathsf{list}\,o \to \Prop \ms\ms\mathbf{by}\\
  &\quad\mathsf{ctxftm}\ms\emptyctx\ms;\\
  &\quad\nabla x. \ms \mathsf{ctxftm}\ms (\istyFh{x} \mathbin{::} L) \eqdef \mathsf{ctxftm}\ms L \ms;\\
  &\quad\nabla x. \ms \mathsf{ctxftm}\ms (\typingFh{x}{A} \mathbin{::} L) \eqdef \mathsf{ctxftm}\ms L \mAnd \{L \vdash \istyFh{A}\}\ms.
\end{align*}
Here $L$ and $A$ are by convention implicitly universally quantified at the outermost level of each rule, that is in particular before the $\nabla x$.
Hence $x$ is fresh for $L$ and $A$ by the nominal axioms of $\mathcal{G}$.
The definition for type formation contexts, \acFty{-}, is even simpler.

Each type in $\mathcal{G}$ is inhabited by a countably infinite number of nominal constants, which are syntactically distinct from each other and all other inhabitants of the respective type.
This allows us to formulate inversion lemmas of the following form:
\begin{align*}
  \acFty{L} \implies \{L \vdash \istyFh{A \to B}\} \implies \{L \vdash \istyFh{A}\} \mAnd \{L \vdash \istyFh{B}\}
\end{align*}
In particular this rules out the case $\istyFh{A \to B} \in L$ which could very well happen without the additional context predicate.
A large part of our present proof relies on formulating the right predicates to suitably constrain contexts, together with appropriate inversion lemmas.
There appears to be some work on automatically generating this infrastructure for basic cases as it is mostly mechanical (CITE LFMTP'14 paper here).
In the following we will have to go a step further and not only require certain contexts to be well-formed for their respective judgements but also to stand in a certain correspondence to other contexts.
Formulating the correct context relation predicates constitutes the heart of the Abella proof presented here.

So let us now look at the actual Abella solution to the equivalence challenge.
We define $\tyr$ and $\tmr$ as follows:
\begin{mathpar}
  \inferrule*{A \tyr a \\ \lpPi {x} B \tyr \lpApp{b}{x}}{A \to B \tyr \Prod a b} \and
  \inferrule*{\lpPi{x y} x \tyr y \lpImp \lpApp{A}{x} \tyr \lpApp{a}{y}}{\All A \tyr \Prod \Prp a}\\
  \inferrule*{s \tmr a \\ t \tmr b}{s\,t \tmr a\,b} \and
  \inferrule*{s \tmr a \\ A \tyr b}{s\,A \tmr a\,b}\\
  \inferrule*{A \tyr a \\ \lpPi{x y} x \tmr y \lpImp \lpApp{s}{x} \tmr \lpApp{b}{y}}{\Lam A s \tmr \Lam a b} \and
  \inferrule*{\lpPi{x y} x \tyr y \lpImp \lpApp{s}{x} \tmr \lpApp{a}{y}}{\TyLam s \tmr \Lam \Prp a}
\end{mathpar}
Note that the techniques used to define these relations are very similar to those used to define the two type systems.
Also note that in contrast to the definition in Coq, no extra parameters/contexts are required to deal with the variable cases, which are themselves implicitly handled via the backchanining rules of the $\lambda$Prolog embedding.
The correct notion of well-formed contexts for these relations, $\ac{\tmr}{~}{-}$, simply consists of pairs of nominals that are either related as types or as terms.
Thus $\ac{\tmr}{~}{L}$ encodes the same information in Abella that was in Coq represented by the two auxiliary variable relations $S$ and $R$.
We do not explicitly define $\ac{\tyr}{~}{-}$ and instead prove the following strengthening result, which holds since derivations of a type relation cannot depend on assumptions about term variables being related.
\begin{align*}
  \ac{\tmr}{~}{L} \implies \{L, x \tmr y \vdash A \tyr a \} \implies \{L \vdash A \tyr a \}
\end{align*}
At this point it is relatively easy to prove that $\tyr$ and $\tmr$ have disjoint ranges and that each is injective and functional.
Apart from several inversion lemmas with respect to context extraction, similar to the one outlined above, the key ingredient for each proof is to show that the employed context $\ac{\tmr}{~}{L}$ already satisfies the respective property.
This in turn is a consequence of the fact that $L$ only contains pairs of related nominals which are, in particular, suitably fresh.

When it comes to the totality and preservation statements, things become more interesting.
Let us consider a first attempt of formulating left totality of $\tyr$ and the preservation of type formation from F to \SysL:
\begin{align*}
  \{L_{F} \vdash \istyFh{A}\} \implies \mEx{a} \{ L_{\tmr} \vdash A \tyr a \} \mAnd \{L_{\lambda} \vdash \typingLh{a}{\Prp}\}
\end{align*}
We will eventually prove the desired result by induction on $\{L_{F} \vdash \istyFh{A}\}$ and it will become quickly apparent that we have to enforce well-formedness of the contexts.
It is however insufficient to simply assume that each of the three contexts is locally well-formed.
While this would ensure that the contexts only contain information about variables (that is only contain nominal assumptions) it misses one key insight: the free variables in $L_{F}$ and those in $L_{\lambda}$ should be associated according to $L_{\tmr}$.
The solution to this is to define a single ternary inductive context predicate that establishes the correspondence.
The freshness assumptions are implemented again with $\nabla$.
\newcommand{\acR}[3]{\ac{R}{~}{#1 \mid #2 \mid #3}}
\begin{mathpar}
  \inferrule*{~}{\acR{\emptyctx}{\emptyctx}{\emptyctx}} \and
  \inferrule*{\acR{L_{F}}{L_{\tmr}}{L_{\lambda}} \\ x, y \notin L_{i}}{\acR{L_{F}, \istyFh{x}}{L_{\tmr}, x \tyr y}{L_{\lambda}, \typingLh{y}{\Prp}}} \and
  \inferrule*{\acR{L_{F}}{L_{\tmr}}{L_{\lambda}} \\ x, y \notin L_{i},A,a \\\\ \{L_{F} \vdash \istyFh{A}\} \\ \{ L_{\tmr} \vdash A \tyr a \} \\ \{L_{\lambda} \vdash \typingLh{a}{\Prp}\}}{\acR{L_{F}, \typingFh{x}{A}}{L_{\tmr}, x \tmr y}{L_{\lambda}, \typingLh{y}{a}}}
\end{mathpar}
It should be clear that $\acR{L_{F}}{L_{\tmr}}{L_{\lambda}}$ entails the local well-formedness of each $L_i$.
Next we establish inversion lemmas that yield for any $J \in L_i$ both the structure of $J$, including the fact that we deal with a nominal, as well as the corresponding $J'$ and $J''$ in the two other contexts.
So to obtain our result we prove the following generalisation and use the outlined inversion principle to handle the variable case:
\begin{align*}
  \{L_{F} \vdash \istyFh{A}\} \implies \mAll{L_{\tmr}L_{\lambda}} \acR{L_{F}}{L_{\tmr}}{L_{\lambda}} \implies \mEx{a} \{ L_{\tmr} \vdash A \tyr a \} \mAnd \{L_{\lambda} \vdash \typingLh{a}{\Prp}\}
\end{align*}
The result for closed judgements and empty contexts is a trivial corollary:
\begin{align*}
  \{\istyFh{A}\} \implies \mEx{a} \{ A \tyr a \} \mAnd \{\typingLh{a}{\Prp}\}
\end{align*}
The $a$ is unique due to functionality of $\tyr$.

The other three equivalences can be obtained with the same ternary predicate.
The usual pattern of first proving results about type-formation and then using them to establish the corresponding results for typing reappears.
It is also again the case that propagation (for both F and \SysL) plays a major role, as well as the degeneracy of the sort $\Typ$.
For the HOAS syntaxes and type systems, however, these properties are relatively easy to obtain.

\section{Beluga}
\label{sec:beluga}

While the HOAS approach is quite elegant in general we were rather unhappy with all the manual manipulation of contextual information that we had to do in Abella.
The need for copious amounts of inversion lemmas that mostly deal with the fact that contexts can a priori contain arbitrary judgements was particularly irritating.
This is what prompted us to try Beluga, an implementation of contextual modal type theory.
In Beluga we do not simply deal with objects (like types or terms) and judgements (like typing), but with so-called contextual entities.
Such entities are, for example, terms that are tagged with contexts that have to cover all the free variables.

So both Abella and Beluga delegate object level-binding to meta-level binding, but while Abella then proceeds to utilises global nominal constants, Beluga instead employs local contexts.
The difference is best illustrated with an example.
Consider, in Abella, the F-function type $n_1 \to n_2$.
It is relatively easy to prove that $\{\emptyctx \vdash \istyFh{n_1 \to n_2}\}$ entails absurdity.
In other words, there exist types $A \OF \TyF$, that are not well-formed.
To be precise, well-formedness checks that a type does not have dangling variables (this is the reason we have made the empty context explicit).

\newcommand{\bc}[2]{\ensuremath{[\,#1\,\vdash #2\,]}}
In Beluga on the other hand, our example does not even exist.
The expression \[\bc{\emptyctx}{X \to Y} \OF \bc{\emptyctx}{\TyF}\] is ill-typed in Beluga's type theory, since $\emptyctx$ does not cover the local variables $X$ and $Y$.
Also observe how the type itself is a contextual object.
An immediate consequence of this is that the concept of a F type formation judgement does not even make sense in Beluga, which will have an impact on the representation of our equivalences.

In contrast to Coq, where contexts are modelled explicitly and are located conceptually on the same level as object language terms and types, and Abella, where the implicit object level reasoning context appears as an explicit list of arbitrary judgements at the reasoning level, Beluga contexts are special reasoning level first class objects that can exhibit rather complex structures.

The definition of our object languages in Beluga is almost identical to the one in Abella.
The only difference is that term abstraction and type application rules of the F type system do not request the respective types are well-formed, as this is immediate for the above reasons.
Hence we won't repeat the formal definitions and instead focus on Beluga's treatment of contextual information.

As mentioned above, Beluga contexts can have a lot of structure.
This is achieved through a separate typing mechanism that ascribes \emph{context schemas} to meta-level variables that represent contexts.
In Beluga, a context is a sequence of dependent records, which each encapsulate multiple pieces of related information.
When we extend such a context with a new record (or block), then all occurring object level entities are taken es contextual objects under the old context.
That is, not only do we have dependency structures within each record but also possibly between multiple records.
A context schema is then a set of admissible $\Sigma$-types for such a list of records.
In particular, this entails that contexts need not be homogeneous.
As a simple example consider the following schema, which is used to prove propagation for \SysL.
\begin{align*}
  S_{\lambda W} \eqdef [x\of\TmL \mrel{;} \typingLh{x}{\Prp}] \mrel{+} [x\of\TmL \mrel{;} \typingLh{x}{a} \mrel{;} \typingLh{a}{\Prp}]
\end{align*}
This schema already separates PTS variables that are used as types (i.e. those from records matching the left variant) and those used as terms (i.e. those from records matching the variant on the right).
Recall that this is a semantic distinction, that is syntactically there is only a single sort.

Let us next consider the proof of \SysL propagation in Beluga, which can be done from first principles and surprisingly is the only language-local meta-theoretic result required to complete the challenge.
We use this to illustrate how a simple inductive Beluga proof is constructed.
The first hurdle we have to overcome is the absence of disjunction and existential quantification in Beluga's meta logic.
Though with a few well-known encoding tricks we can still define a predicate $\mathsf{type\_correct}\,a$ with suitable constructors, $p_{\mathsf{tc1}}$ and $p_{\mathsf{tc2}}$, satisfying:
\begin{align*}
  \mathsf{type\_correct}\,a \iff a = \Typ \mOr \mEx{u} \typingLh{a}{u} \mAnd \sortLh{u}
\end{align*}
Since Beluga is not equipped with a tactic language, we prove propagation by implementing a total recursive function of the following type:
\begin{align*}
  \mAll{\Gamma \of S_{\lambda W}} \bc{\Gamma}{\typingLh{A}{B}} \implies \bc{\Gamma}{\mathsf{type\_correct}\,B}
\end{align*}
The proof proceeds by discriminating on $d \OF \bc{\Gamma}{\typingLh{A}{B}}$ which can take one of 7 forms, 3 of which are context extractions.
Let us first look at the 4 structural cases.
When $d$ is a derivation of $\bc{\Gamma}{\typingLh{\Prp}{\Typ}}$ then $\bc{\Gamma}{\mathsf{type\_correct}\,\Typ}$ clearly holds.
For product formation, $\bc{\Gamma}{\mathsf{type\_correct}\,\Prp}$ is easily established and for the abstraction rule we have to show that $\bc{\Gamma}{\typingLh{\Prod{A}B}{\Prp}}$ holds, and thus $\bc{\Gamma}{\mathsf{type\_correct}\,\Prod{A}B}$, which is easy.
The interesting case is application.
That is we have $d = \bc{\Gamma}{r_{\mathsf{app}}\,d_{AB}\,d_{A}} \OF \bc{\Gamma}{\typingLh{m\,n}{\lpApp{B}{n}}}$, where $d_{AB} \OF \bc{\Gamma}{\typingLh{M}{\Prod{A}B}}$ and $d_{A} \OF \bc{\Gamma}{\typingLh{N}{A}}$.
Applying our inductive hypothesis to $\bc{\Gamma}{d_{AB}}$ yields $\bc{\Gamma}{p_{\mathsf{tc2}}\,d_P\,d_u} \OF \bc{\Gamma}{\mathsf{type\_correct}\,\Prod{A}B}$.
Note that higher order unification is able to discharge the alternative $\bc{\Gamma}{p_{\mathsf{tc1}}} \OF \bc{\Gamma}{\mathsf{type\_correct}\,\Prod{A}B}$ since $\Prod{A}B \neq \Typ$.
Further analysis of $d_P$ reveals a derivation $d_B \OF \bc{\Gamma, [\mathsf{istml}\,x \mrel{;} d_x \of \typingLh{x}{A}]}{\typingLh{\lpApp{B}{x}}{\Prp}}$.
Substitutivity is natively provided in Beluga, so we have $d_B\subst{N,d_A} \OF \bc{\Gamma}{\typingLh{\lpApp{B}{N}}{\Prp}}$ and can close the case.
This brings us to context extraction where we recall the definition of $S_{\lambda W}$.
The left alternative has one occurrence of a typing assumption, while the right alternative has two.
Two of these fix $B = \Prp$ so the requisite universe containing $B$ is simply $\Typ$.
If however we happen to have extracted the first typing assumption from a record of the right variant, then $S_{\lambda W}$ tells us that the second typing assumption from the very same block assigns $B$ the type $\Prp$, which is the required universe to close the case.

The most interesting part here was the use of substitutivity, which in Coq required a significant amount of boilerplate development.
And even in Abella we first had to obtain substitutivity of PTS typing as a special case of a cut theorem and combine this with the aforementioned context inversion Lemmas.

At this point we can tackle the actual solution to our challenge.
Both $\tyr$ and $\tmr$ again encoded as LF types at the same level where the object languages are defined.
The definitions are exactly identical to those in Abella.
Since Beluga does not natively provide equality, we also define three LF types that encode equality of the various syntactic sorts, each with a single reflexive constructor.

We begin with the functionality of $\tyr$, that is, we implement a recursive function of type
\begin{align*}
  \mAll{\Gamma \of S_{\tyr}} \bc{\Gamma}{A \tyr a} \implies \bc{\Gamma}{A \tyr a'} \implies \bc{\Gamma}{a =_{\lambda} a'},
\end{align*}
where
\begin{align*}
  S_{\tyr} \eqdef [y \of \TmL] \mrel{+} [x\of\TyF \mrel{;} y\of\TmL \mrel{;} h\of x \tyr y]
\end{align*}
We go over the definition of this function in Detail to illustrate how contextual information is generated and processed when binders are traversed.
At a high level, the proof is by induction on the derivation of $d_1 \OF \bc{\Gamma}{A \tyr a}$ and a discrimination on $d_2 \OF \bc{\Gamma}{A \tyr a'}$.
We have to cover three cases, two structural and one for a context extraction.

Let $d_1$ be a derivation that ends in the application of the arrow rule, that is $A = B \to C$ and $a = \Prod{b}c$ for some $B,C,b,c$.
We then have sub-derivations $d_{11} \OF \bc{\Gamma}{B \tyr b}$ and $d_{12} \OF \bc{\Gamma, [y \of \TmL]}{C \tyr \lpApp{c}{y}}$.
As a consequence, $d_2$ must have ended with the arrow rule as well, whence $a' = \Prod{b'}c'$, with sub-derivations $d_{21} \OF \bc{\Gamma}{B \tyr b'}$ and $d_{22} \OF \bc{\Gamma, [y \of \TmL]}{C \tyr \lpApp{c'}{y}}$.
We have to prove that $\Prod{b}c =_{\lambda} \Prod{b'}c'$, which is only possible if we can unify $b$ with $b'$ and $c$ with $c'$.
This is easily achieved through invocations of the inductive hypothesis on $d_{11}$ and $d_{21}$ and, respectively, on $d_{12}$ and $d_{22}$.

Next we consider the product rule, that is $A = \All B$, $a = \Prod{\Prp} b$ and $a' = \Prod{\Prp} b'$.
When we consider the sub-derivation $d_{11} \OF \bc{\Gamma, x \of \TyF, y \of \TmL, h \of x \tyr y}{\lpApp{B}{x} \tyr \lpApp{b}{y}}$ we notice that the context does no longer satisfy our schema, as the new assumptions are not appropriately packaged (Q-at-BP: why does this happen/matter \& could it be avoided somehow?)
The scenario is similar for $d_{21}$.
That is, we cannot directly apply our inductive hypothesis to unify $b$ and $b'$.
The trick is to generate a fresh record $r \OF [x \of \TyF, y \of \TmL, h \of x \tyr y]$  and then invoke the induction hypothesis on $\bc{\Gamma, r}{d_{11}\subst{r_x,r_y,r_h}}$ and $\bc{\Gamma, r}{d_{21}\subst{r_x,r_y,r_h}}$.

Finally, for the extraction case, the derivation $d_1$ must have come from a block of the form $r \OF [x\of\TyF \mrel{;} y\of\TmL \mrel{;} h \of x \tyr y]$, that is $d_1 = r_h$.
Then $d_2 = r'_h$ for some other record $r'$, but we also have $r_x = r'_x$ and we further know that each $x$ is local and unique to its containing $r$, resp.\ $r'$, (Q-at-BP: Why exactly?).
The only solution to these constraints is $r = r'$, and thus in particular $r_y = r'_y$, closing the case.

We can use $S_{\tyr}$ to obtain injectivity of $\tyr$ in a similar fashion.
To obtain functionality and injectivity of $\tmr$ we can follow a similar pattern, albeit with a different context schema
\begin{align*}
  S_{\tmr} \eqdef [x\of\TmF \mrel{;} y\of\TmL \mrel{;} h\of x \tmr y] \mrel{+} [x\of\TyF \mrel{;} y\of\TmL \mrel{;} h\of x \tyr y]
\end{align*}
When we compare the two context schemas it should quickly be come apparent that any $\Gamma \of S_{\tmr}$ contains at least as much information as required by $S_{\tyr}$.
Beluga is capable of implicitly strengthening contexts in this way.
So when the proof of functionality of $\tmr$ relies on the functionality of $\tyr$ we can directly pass the given complex context and do not need to invoke any stripping functions or strengthening lemmas beforehand.
In other words, we get directly $\Gamma \of S_{\tyr}$ rather than $\mathsf{str}(\Gamma) \of S_{\tyr}$.
The only other interesting aspect here than is the injectivity of $\tmr$ relies on $\tyr$ and $\tmr$ having disjoint ranges, which thanks to higher-order unification is very easy to obtain under contexts $\Gamma \of S_{\tmr}$.

When it comes to the totality and preservation results we again have to use encoding tricks to represent the existential nature of the goal.
We define, for example, a predicate that codes the existence of a related PTS proposition for a given F type:
\begin{align*}
  \mathsf{exists\_rel\_prop}\,A \iff \mEx{a} A \tyr a \mAnd \typingLh{a}{\Prp}
\end{align*}
We also have to extend $S_{\tyr}$ with additional typing information in each block:
\begin{align*}
  S_{\tyr W}^{\rightarrow} \eqdef [y\of\TmL \mrel{;} v\of\typingLh{y}{a}] \mrel{+} [x\of\TyF \mrel{;} y\of\TmL \mrel{;} h\of x \tyr y \mrel{;} v\of\typingLh{y}{\Prp}]
\end{align*}
The result itself is then established with a function of type
\begin{align*}
  \mAll{\Gamma \of S_{\tyr W}^{\rightarrow}} \mAll{A \of \bc{\Gamma}{\TyF}} \bc{\Gamma}{\mathsf{exists\_rel\_prop}\,A}
\end{align*}
We are not going to present its construction in detail.
The interesting part is due to the fact that we cannot have an object level type formation judgement for Beluga's variant of F types.
The key part here is the typing $A \of \bc{\Gamma}{\TyF}$ which is itself contextual.
Since every type is well-formed by construction, we prove this result immediately by induction on $A$.
Recall that this was not possible in the Abella proof, where the well-formedness required a separate predicate that in turn was essential to carry the induction through.

For similar reasons the encoding of the goal for the inverse direction only yields
\begin{align*}
  \mathsf{exists\_rel\_type}\,a \iff \mEx{A} A \tyr a,
\end{align*}
that is, if there is some $A$ at all that is related via $\tyr$ then $A$ is well-formed by construction.
Interestingly, it appears that we need even more typing information in the left block of the schema in order to prove the respective Lemma\footnote{There are reasons to believe that the same schema should  in theory work for both directions, but we have not yet investigated this further.}:
\begin{align*}
  S_{\tyr W}^{\leftarrow} \eqdef &[y\of\TmL \mrel{;} v\of\typingLh{y}{a} \mrel{;} h\of A \tyr a \mrel{;} w \of \typingLh{a}{\Prp}]\\
  \mrel{+} &[x\of\TyF \mrel{;} y\of\TmL \mrel{;} h\of x \tyr y \mrel{;} v\of\typingLh{y}{\Prp}]
\end{align*}

The final two results are left and right preservation and totality of $\tmr$.
Both can be established with suitable encodings of the existentials and the following schema that interstingly works for both directions:\footnote{This is the reason we conjecture that for $\tyr$ the same should be doable.}
\begin{align*}
  S_{\tmr W} \eqdef &[x\of\TmF \mrel{;} y\of\TmL \mrel{;} h\of x \tmr y \mrel{;} u\of\typingFh{x}{A} \mrel{;} v\of\typingLh{y}{a} \mrel{;} j \of A \tyr a]\\
  \mrel{+} &[x\of\TyF \mrel{;} y\of\TmL \mrel{;} h\of x \tyr y \mrel{;} v\of\typingLh{y}{\Prp}]
\end{align*}

Once this schema has been fixed, constructing the two proofs is rather technical and lengthy, but not inherently different from the proofs in Coq or Abella.
Note that any $\Gamma \of S_{\tmr A}$ also matches both $S_{\tyr W}^{\rightarrow}$ and $S_{\tyr W}^{\leftarrow}$ so that corresponding results for type formation can easily be invoked.

\section{Discussion and Observations}
\label{sec:disc-observ}

When we have to construct a proof for an intricate result, we are often tempted to just grab our favourite proof assistant and start hacking away.
While the approach may, with sufficient expertise, lead to a technically correct proof it is often the case that the end product does not reveal much more about the problem than ``the result holds''.
If we allow ourselves a moment of philosophical musing than we may recall that it is often claimed that the main purpose of a proof is to communicate, and convince, someone else that a certain fact follows from mutually accepted assumptions.
In this regard, all we have achieved so far is to convince some machine of such a fact.
In order to communicate the result to our colleagues as well, we usually need more than just a proof script or proof term -- we need intuitions.


\section{Conclusion and Outlook}
\label{sec:conclusion}

\subsection{Future Work}
\label{sec:future-work}

The results and observations obtained here are already interesting in their own right.
As a benchmark, however, they are only of a preliminary nature.
Thus we would like to extend the present work in at least two directions.

Firstly we would like to cover additional frameworks in order to see how well these are able to handle the present challenge.
Those of immediate interest are the so-called locally nameless techniques~\cite{DBLP:conf/popl/AydemirCPPW08} and the HYBRID framework~\cite{Capretta2007, Capretta2009, DBLP:journals/jar/FeltyM12}.
The locally nameless approach provides an abstraction layer that can be seen as sitting between the pure and low-level de Bruijn approach on the one hand and the high-level HOAS approach on the other.
The HYBRID framework for Coq, on the other hand, is in spirit very close to Abella and provides a HOAS layer for Coq.
It does, however, lack Abella's $\nabla$ and neither is it equipped with Beluga's capabilities of fine-grained contextual reasoning, so it remains to be seen if the framework is capable of dealing with our benchmark.
Should it turn out that HYBRID is not (yet) up to the task, then our benchmark could act as a useful guideline for future development.

The other direction we would like to focus on concerns the benchmark itself.
At the moment, we only look at the equivalence of the typability problem.
Note, however, that we are dealing with computational systems, and as such their reduction behaviours are at least as interesting as their typability problems.
Equi-reducebility will likely present its own set of challenges, as the PTS has many more $\beta$-redices, at least prior to typing.
And reduction is of course usually defined independent of typing.

Finally it would be interesting to formulate the whole setup not only for System~F, but also for the simply typed $\lambda$-calculus on the one hand and F$_\omega$ on the other.
For the remaining corners of Barendregt's $\lambda$-cube, like for example the calculus of constructions, no corresponding two-sorted variants exist.
It is thus unclear what a suitable benchmark would look like.


% \subparagraph*{Acknowledgements.}

% I want to thank \dots

% \appendix
% \section{FOO}

% Needed?



%%
%% Bibliography
%%

\bibliography{ref}

\end{document}

%%% Local Variables:
%%% mode: latex
%%% TeX-master: t
%%% End:
